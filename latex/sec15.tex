\section{Tfe main program}\label{tfe-main-program}

The file \passthrough{\lstinline!tfeapplication.c!} is a main program of
Tfe. It includes all the code other than
\passthrough{\lstinline!tfetextview.c!} and
\passthrough{\lstinline!tfenotebook.c!}. It does:

\begin{itemize}
\tightlist
\item
  Application support, mainly handling command line arguments.
\item
  Builds widgets using ui file.
\item
  Connects button signals and their handlers.
\item
  Manages CSS.
\end{itemize}

\subsection{The function main}\label{the-function-main}

The function \passthrough{\lstinline!main!} is the first invoked
function in C language. It connects the command line given by the user
and Gtk application.

\begin{lstlisting}[language=C, numbers=left]
#define APPLICATION_ID "com.github.ToshioCP.tfe"

int
main (int argc, char **argv) {
  GtkApplication *app;
  int stat;

  app = gtk_application_new (APPLICATION_ID, G_APPLICATION_HANDLES_OPEN);

  g_signal_connect (app, "startup", G_CALLBACK (app_startup), NULL);
  g_signal_connect (app, "activate", G_CALLBACK (app_activate), NULL);
  g_signal_connect (app, "open", G_CALLBACK (app_open), NULL);

  stat =g_application_run (G_APPLICATION (app), argc, argv);
  g_object_unref (app);
  return stat;
}
\end{lstlisting}

\begin{itemize}
\tightlist
\item
  1: Defines the application id. Thanks to the
  \passthrough{\lstinline!\#define!} directive, it is easy to find the
  application id.
\item
  8: Creates GtkApplication object.
\item
  10-12: Connects ``startup'', ``activate'' and ``open'' signals to
  their handlers.
\item
  14: Runs the application.
\item
  15-16: Releases the reference to the application and returns the
  status.
\end{itemize}

\subsection{Startup signal handler}\label{startup-signal-handler}

Startup signal is emitted just after the GtkApplication instance is
initialized. The handler initializes the whole application which
includes not only GtkApplication instance but also widgets and some
other objects.

\begin{itemize}
\tightlist
\item
  Builds the widgets using ui file.
\item
  Connects button signals and their handlers.
\item
  Sets CSS.
\end{itemize}

The handler is as follows.

\begin{lstlisting}[language=C, numbers=left]
static void
app_startup (GApplication *application) {
  GtkApplication *app = GTK_APPLICATION (application);
  GtkBuilder *build;
  GtkApplicationWindow *win;
  GtkNotebook *nb;
  GtkButton *btno;
  GtkButton *btnn;
  GtkButton *btns;
  GtkButton *btnc;

  build = gtk_builder_new_from_resource ("/com/github/ToshioCP/tfe/tfe.ui");
  win = GTK_APPLICATION_WINDOW (gtk_builder_get_object (build, "win"));
  nb = GTK_NOTEBOOK (gtk_builder_get_object (build, "nb"));
  gtk_window_set_application (GTK_WINDOW (win), app);
  btno = GTK_BUTTON (gtk_builder_get_object (build, "btno"));
  btnn = GTK_BUTTON (gtk_builder_get_object (build, "btnn"));
  btns = GTK_BUTTON (gtk_builder_get_object (build, "btns"));
  btnc = GTK_BUTTON (gtk_builder_get_object (build, "btnc"));
  g_signal_connect_swapped (btno, "clicked", G_CALLBACK (open_cb), nb);
  g_signal_connect_swapped (btnn, "clicked", G_CALLBACK (new_cb), nb);
  g_signal_connect_swapped (btns, "clicked", G_CALLBACK (save_cb), nb);
  g_signal_connect_swapped (btnc, "clicked", G_CALLBACK (close_cb), nb);
  g_object_unref(build);

GdkDisplay *display;

  display = gdk_display_get_default ();
  GtkCssProvider *provider = gtk_css_provider_new ();
  gtk_css_provider_load_from_data (provider, "textview {padding: 10px; font-family: monospace; font-size: 12pt;}", -1);
  gtk_style_context_add_provider_for_display (display, GTK_STYLE_PROVIDER (provider), GTK_STYLE_PROVIDER_PRIORITY_APPLICATION);

  g_signal_connect (win, "destroy", G_CALLBACK (before_destroy), provider);
  g_object_unref (provider);
}
\end{lstlisting}

\begin{itemize}
\tightlist
\item
  12-15: Builds widgets using ui resource. Connects the top-level window
  and the application with
  \passthrough{\lstinline!gtk\_window\_set\_application!}.
\item
  16-23: Gets buttons and connects their signals and handlers. The macro
  \passthrough{\lstinline!g\_signal\_connect\_swapped!} connects a
  signal and handler like \passthrough{\lstinline!g\_signal\_connect!}.
  The difference is that
  \passthrough{\lstinline!g\_signal\_connect\_swapped!} swaps the user
  data for the object. For example, the macro on line 20 swaps
  \passthrough{\lstinline!nb!} for \passthrough{\lstinline!btno!}. So,
  the handler expects that the first argument is
  \passthrough{\lstinline!nb!} instead of
  \passthrough{\lstinline!btno!}.
\item
  24: Releases the reference to GtkBuilder.
\item
  26-31: Sets CSS. CSS in Gtk is similar to CSS in HTML. You can set
  margin, border, padding, color, font and so on with CSS. In this
  program, CSS is on line 30. It sets padding, font-family and font size
  of GtkTextView. CSS will be explained in the next subsection.
\item
  26-28: GdkDisplay is used to set CSS. The default GdkDisplay object
  can be obtain with the function
  \passthrough{\lstinline!gfk\_display\_get\_default!}. This function
  needs to be called after the window creation.
\item
  33: Connects ``destroy'' signal on the main window and before\_destroy
  handler. This handler is explained in the next subsection.
\item
  34: The provider is useless for the startup handler, so it is
  released. Note: It doesn't mean the destruction of the provider. It is
  referred by the display so the reference count is not zero.
\end{itemize}

\subsection{CSS in Gtk}\label{css-in-gtk}

CSS is an abbreviation of Cascading Style Sheet. It is originally used
with HTML to describe the presentation semantics of a document. You
might have found that widgets in Gtk is similar to elements in HTML. It
tells that CSS can be applied to Gtk windowing system, too.

\subsubsection{CSS nodes, selectors}\label{css-nodes-selectors}

The syntax of CSS is as follows.

\begin{lstlisting}
selector { color: yellow; padding-top: 10px; ...}
\end{lstlisting}

Every widget has CSS node. For example, GtkTextView has
\passthrough{\lstinline!textview!} node. If you want to set style to
GtkTextView, substitute ``textview'' for
\passthrough{\lstinline!selector!} above.

\begin{lstlisting}
textview {color: yellow; ...}
\end{lstlisting}

Class, ID and some other things can be applied to the selector like Web
CSS. Refer to \href{https://docs.gtk.org/gtk4/css-overview.html}{GTK 4
API Reference -- CSS in Gtk} for further information.

The codes of the startup handler has a CSS string on line 30.

\begin{lstlisting}
textview {padding: 10px; font-family: monospace; font-size: 12pt;}
\end{lstlisting}

\begin{itemize}
\tightlist
\item
  Padding is a space between the border and contents. This space makes
  the textview easier to read.
\item
  font-family is a name of font. The font name ``monospace'' is one of
  the generic family font keywords.
\item
  Font-size is set to 12pt.
\end{itemize}

\subsubsection{GtkStyleContext, GtkCssProvider and
GdkDisplay}\label{gtkstylecontext-gtkcssprovider-and-gdkdisplay}

GtkStyleContext is deprecated since version 4.10. But two functions
\passthrough{\lstinline!gtk\_style\_context\_add\_provider\_for\_display!}
and
\passthrough{\lstinline!gtk\_style\_context\_remove\_provider\_for\_display!}
are not deprecated. They add or remove a css provider object to the
GdkDisplay object.

GtkCssProvider is an object which parses CSS for style widgets.

To apply your CSS to widgets, you need to add GtkStyleProvider (the
interface of GtkCssProvider) to the GdkDisplay object. You can get the
default display object with the function
\passthrough{\lstinline!gdk\_display\_get\_default!}. The returned
object is owned by the function and you don't have its ownership. So,
you don't need to care about releasing it.

Look at the source file of \passthrough{\lstinline!startup!} handler
again.

\begin{itemize}
\tightlist
\item
  28: The display is obtained by
  \passthrough{\lstinline!gdk\_display\_get\_default!}.
\item
  29: Creates a GtkCssProvider instance.
\item
  30: Puts the CSS into the provider. The function
  \passthrough{\lstinline!gtk\_css\_provider\_load\_from\_data!} will be
  deprecated since 4.12 (Not 4.10). The new function
  \passthrough{\lstinline!gtk\_css\_provider\_load\_from\_string!} will
  be used in the future version of Tfe.
\item
  31: Adds the provider to the display. The last argument of
  \passthrough{\lstinline!gtk\_style\_context\_add\_provider\_for\_display!}
  is the priority of the style provider.
  \passthrough{\lstinline!GTK\_STYLE\_PROVIDER\_PRIORITY\_APPLICATION!}
  is a priority for application-specific style information. Refer to
  \href{https://docs.gtk.org/gtk4/index.html\#constants}{GTK 4 Reference
  --- Constants} for more information. You can find other constants,
  which have ``STYLE\_PROVIDER\_PRIORITY\_XXXX'' pattern names.
\end{itemize}

\begin{lstlisting}[language=C, numbers=left]
static void
before_destroy (GtkWidget *win, GtkCssProvider *provider) {
  GdkDisplay *display = gdk_display_get_default ();
  gtk_style_context_remove_provider_for_display (display, GTK_STYLE_PROVIDER (provider));
}
\end{lstlisting}

When a widget is destroyed, or more precisely during its disposing
process, a ``destroy'' signal is emitted. The ``before\_destroy''
handler connects to the signal on the main window. (See the program list
of app\_startup.) So, it is called when the window is destroyed.

The handler removes the CSS provider from the GdkDisplay.

Note: CSS providers are removed automatically when the application
quits. So, even if the handler \passthrough{\lstinline!before\_destroy!}
is removed, the application works.

\subsection{Activate and open signal
handler}\label{activate-and-open-signal-handler}

The handlers of ``activate'' and ``open'' signals are
\passthrough{\lstinline!app\_activate!} and
\passthrough{\lstinline!app\_open!} respectively. They just create a new
GtkNotebookPage.

\begin{lstlisting}[language=C, numbers=left]
static void
app_activate (GApplication *application) {
  GtkApplication *app = GTK_APPLICATION (application);
  GtkWidget *win = GTK_WIDGET (gtk_application_get_active_window (app));
  GtkWidget *boxv = gtk_window_get_child (GTK_WINDOW (win));
  GtkNotebook *nb = GTK_NOTEBOOK (gtk_widget_get_last_child (boxv));

  notebook_page_new (nb);
  gtk_window_present (GTK_WINDOW (win));
}

static void
app_open (GApplication *application, GFile ** files, gint n_files, const gchar *hint) {
  GtkApplication *app = GTK_APPLICATION (application);
  GtkWidget *win = GTK_WIDGET (gtk_application_get_active_window (app));
  GtkWidget *boxv = gtk_window_get_child (GTK_WINDOW (win));
  GtkNotebook *nb = GTK_NOTEBOOK (gtk_widget_get_last_child (boxv));
  int i;

  for (i = 0; i < n_files; i++)
    notebook_page_new_with_file (nb, files[i]);
  if (gtk_notebook_get_n_pages (nb) == 0)
    notebook_page_new (nb);
  gtk_window_present (GTK_WINDOW (win));
}
\end{lstlisting}

\begin{itemize}
\tightlist
\item
  1-10: \passthrough{\lstinline!app\_activate!}.
\item
  8-10: Creates a new page and shows the window.
\item
  12-25: \passthrough{\lstinline!app\_open!}.
\item
  20-21: Creates notebook pages with files.
\item
  22-23: If no page has created, maybe because of read error, then it
  creates an empty page.
\item
  24: Shows the window.
\end{itemize}

These codes have become really simple thanks to tfenotebook.c and
tfetextview.c.

\subsection{A primary instance}\label{a-primary-instance}

Only one GApplication instance can be run at a time in a session. The
session is a bit difficult concept and also platform-dependent, but
roughly speaking, it corresponds to a graphical desktop login. When you
use your PC, you probably login first, then your desktop appears until
you log off. This is the session.

However, Linux is multi process OS and you can run two or more instances
of the same application. Isn't it a contradiction?

When first instance is launched, then it registers itself with its
application ID (for example,
\passthrough{\lstinline!com.github.ToshioCP.tfe!}). Just after the
registration, startup signal is emitted, then activate or open signal is
emitted and the instance's main loop runs. I wrote ``startup signal is
emitted just after the application instance is initialized'' in the
prior subsection. More precisely, it is emitted after the registration.

If another instance which has the same application ID is launched, it
also tries to register itself. Because this is the second instance, the
registration of the ID has already done, so it fails. Because of the
failure startup signal isn't emitted. After that, activate or open
signal is emitted in the primary instance, not on the second instance.
The primary instance receives the signal and its handler is invoked. On
the other hand, the second instance doesn't receive the signal and it
immediately quits.

Try running two instances in a row.

\begin{lstlisting}
$ ./_build/tfe &
[1] 84453
$ ./build/tfe tfeapplication.c
$
\end{lstlisting}

First, the primary instance opens a window. Then, after the second
instance is run, a new notebook page with the contents of
\passthrough{\lstinline!tfeapplication.c!} appears in the primary
instance's window. This is because the open signal is emitted in the
primary instance. The second instance immediately quits so shell prompt
soon appears.

\subsection{A series of handlers correspond to the button
signals}\label{a-series-of-handlers-correspond-to-the-button-signals}

\begin{lstlisting}[language=C, numbers=left]
static void
open_cb (GtkNotebook *nb) {
  notebook_page_open (nb);
}

static void
new_cb (GtkNotebook *nb) {
  notebook_page_new (nb);
}

static void
save_cb (GtkNotebook *nb) {
  notebook_page_save (nb);
}

static void
close_cb (GtkNotebook *nb) {
  notebook_page_close (GTK_NOTEBOOK (nb));
}
\end{lstlisting}

\passthrough{\lstinline!open\_cb!}, \passthrough{\lstinline!new\_cb!},
\passthrough{\lstinline!save\_cb!} and
\passthrough{\lstinline!close\_cb!} just call corresponding notebook
page functions.

\subsection{meson.build}\label{meson.build}

\begin{lstlisting}[numbers=left]
project('tfe', 'c')

gtkdep = dependency('gtk4')

gnome=import('gnome')
resources = gnome.compile_resources('resources','tfe.gresource.xml')

sourcefiles=files('tfeapplication.c', 'tfenotebook.c', '../tfetextview/tfetextview.c')

executable('tfe', sourcefiles, resources, dependencies: gtkdep)
\end{lstlisting}

In this file, just the source file names are modified from the prior
version.

\subsection{source files}\label{source-files}

You can download the files from the
\href{https://github.com/ToshioCP/Gtk4-tutorial}{repository}. There are
two options.

\begin{itemize}
\tightlist
\item
  Use git and clone.
\item
  Run your browser and open the
  \href{https://github.com/ToshioCP/Gtk4-tutorial}{top page}. Then click
  on ``Code'' button and click ``Download ZIP'' in the popup menu. After
  that, unzip the archive file.
\end{itemize}

If you use git, run the terminal and type the following.

\begin{lstlisting}
$ git clone https://github.com/ToshioCP/Gtk4-tutorial.git
\end{lstlisting}

The source files are under /src/tfe5 directory.
