\section{Periodic Events}\label{periodic-events}

This chapter was written by Paul Schulz
\href{mailto:paul@mawsonlakes.org}{\nolinkurl{paul@mawsonlakes.org}}.

\subsection{How do we create an
animation?}\label{how-do-we-create-an-animation}

In this section we will continue to build on our previous work. We will
create an analog clock application. By adding a function which
periodically redraws GtkDrawingArea, the clock will be able to
continuously display the time.

The application uses a compiled in `resource' file, so if the GTK4
libraries and their dependencies are installed and available, the
application will run from anywhere.

The program also makes use of some standard mathematical and time
handling functions.

The clocks mechanics were taken from a Cairo drawing example, using
gtkmm4, which can be found
\href{https://developer-old.gnome.org/gtkmm-tutorial/stable/sec-drawing-clock-example.html.en}{here}.

The complete code is at the end.

\subsection{Drawing the clock face, hour, minute and second
hands}\label{drawing-the-clock-face-hour-minute-and-second-hands}

The \passthrough{\lstinline!draw\_clock()!} function does all the work.
See the in-file comments for an explanation of how the Cairo drawing
works.

For a detailed reference of what each of the Cairo functions does see
the
\href{https://www.cairographics.org/manual/cairo-cairo-t.html}{cairo\_t
reference}.

\begin{lstlisting}[language=C, numbers=left]
static void
draw_clock (GtkDrawingArea *area, cairo_t *cr, int width, int height, gpointer user_data) {

    // Scale to unit square and translate (0, 0) to be (0.5, 0.5), i.e.
    // the center of the window
    cairo_scale(cr, width, height);
    cairo_translate(cr, 0.5, 0.5);

    // Set the line width and save the cairo drawing state.
    cairo_set_line_width(cr, m_line_width);
    cairo_save(cr);

    // Set the background to a slightly transparent green.
    cairo_set_source_rgba(cr, 0.337, 0.612, 0.117, 0.9);   // green
    cairo_paint(cr);

    // Resore back to precious drawing state and draw the circular path
    // representing the clockface. Save this state (including the path) so we
    // can reuse it.
    cairo_restore(cr);
    cairo_arc(cr, 0.0, 0.0, m_radius, 0.0, 2.0 * M_PI);
    cairo_save(cr);

    // Fill the clockface with white
    cairo_set_source_rgba(cr, 1.0, 1.0, 1.0, 0.8);
    cairo_fill_preserve(cr);
    // Restore the path, paint the outside of the clock face.
    cairo_restore(cr);
    cairo_stroke_preserve(cr);
    // Set the 'clip region' to the inside of the path (fill region).
    cairo_clip(cr);

    // Clock ticks
    for (int i = 0; i < 12; i++)
    {
        // Major tick size
        double inset = 0.05;

        // Save the graphics state, restore after drawing tick to maintain pen
        // size
        cairo_save(cr);
        cairo_set_line_cap(cr, CAIRO_LINE_CAP_ROUND);

        // Minor ticks are shorter, and narrower.
        if(i % 3 != 0)
        {
            inset *= 0.8;
            cairo_set_line_width(cr, 0.03);
        }

        // Draw tick mark
        cairo_move_to(
            cr,
            (m_radius - inset) * cos (i * M_PI / 6.0),
            (m_radius - inset) * sin (i * M_PI / 6.0));
        cairo_line_to(
            cr,
            m_radius * cos (i * M_PI / 6.0),
            m_radius * sin (i * M_PI / 6.0));
        cairo_stroke(cr);
        cairo_restore(cr); /* stack-pen-size */
    }

    // Draw the analog hands

    // Get the current Unix time, convert to the local time and break into time
    // structure to read various time parts.
    time_t rawtime;
    time(&rawtime);
    struct tm * timeinfo = localtime (&rawtime);

    // Calculate the angles of the hands of our clock
    double hours   = timeinfo->tm_hour * M_PI / 6.0;
    double minutes = timeinfo->tm_min * M_PI / 30.0;
    double seconds = timeinfo->tm_sec * M_PI / 30.0;

    // Save the graphics state
    cairo_save(cr);
    cairo_set_line_cap(cr, CAIRO_LINE_CAP_ROUND);

    cairo_save(cr);

    // Draw the seconds hand
    cairo_set_line_width(cr, m_line_width / 3.0);
    cairo_set_source_rgba(cr, 0.7, 0.7, 0.7, 0.8);   // gray
    cairo_move_to(cr, 0.0, 0.0);
    cairo_line_to(cr,
                  sin(seconds) * (m_radius * 0.9),
                  -cos(seconds) * (m_radius * 0.9));
    cairo_stroke(cr);
    cairo_restore(cr);

    // Draw the minutes hand
    cairo_set_source_rgba(cr, 0.117, 0.337, 0.612, 0.9);   // blue
    cairo_move_to(cr, 0, 0);
    cairo_line_to(cr,
                  sin(minutes + seconds / 60) * (m_radius * 0.8),
                  -cos(minutes + seconds / 60) * (m_radius * 0.8));
    cairo_stroke(cr);

    // draw the hours hand
    cairo_set_source_rgba(cr, 0.337, 0.612, 0.117, 0.9);   // green
    cairo_move_to(cr, 0.0, 0.0);
    cairo_line_to(cr,
                  sin(hours + minutes / 12.0) * (m_radius * 0.5),
                  -cos(hours + minutes / 12.0) * (m_radius * 0.5));
    cairo_stroke(cr);
    cairo_restore(cr);

    // Draw a little dot in the middle
    cairo_arc(cr, 0.0, 0.0, m_line_width / 3.0, 0.0, 2.0 * M_PI);
    cairo_fill(cr);
}
\end{lstlisting}

In order for the clock to be drawn, the drawing function
\passthrough{\lstinline!draw\_clock()!} needs to be registered with
GTK4. This is done in the \passthrough{\lstinline!app\_activate()!}
function (on line 24).

Whenever the application needs to redraw the GtkDrawingArea, it will now
call \passthrough{\lstinline!draw\_clock()!}.

There is still a problem though. In order to animate the clock we need
to also tell the application that the clock needs to be redrawn every
second. This process starts by registering (on the next line, line 15) a
timeout function with \passthrough{\lstinline!g\_timeout\_add()!} that
will wakeup and run another function
\passthrough{\lstinline!time\_handler!}, every second (or 1000ms).

\begin{lstlisting}[language=C, numbers=left]
static void
app_activate (GApplication *app, gpointer user_data) {
    GtkWidget *win;
    GtkWidget *clock;
    GtkBuilder *build;

    build = gtk_builder_new_from_resource ("/com/github/ToshioCP/tfc/tfc.ui");
    win = GTK_WIDGET (gtk_builder_get_object (build, "win"));
    gtk_window_set_application (GTK_WINDOW (win), GTK_APPLICATION (app));

    clock = GTK_WIDGET (gtk_builder_get_object (build, "clock"));
    g_object_unref(build);

    gtk_drawing_area_set_draw_func(GTK_DRAWING_AREA (clock), draw_clock, NULL, NULL);
    g_timeout_add(1000, (GSourceFunc) time_handler, (gpointer) clock);
    gtk_window_present(GTK_WINDOW (win));

}
\end{lstlisting}

Our \passthrough{\lstinline!time\_handler()!} function is very simple,
as it just calls \passthrough{\lstinline!gtk\_widget\_queue\_draw()!}
which schedules a redraw of the widget.

\begin{lstlisting}[language=C, numbers=left]
gboolean
time_handler(GtkWidget* widget) {
    gtk_widget_queue_draw(widget);

    return TRUE;
}
\end{lstlisting}

.. and that is all there is to it. If you compile and run the example
you will get a ticking analog clock.

If you get this working, you can try modifying some of the code in
\passthrough{\lstinline!draw\_clock()!} to tweak the application (such
as change the color or size and length of the hands) or even add text,
or create a digital clock.

\subsection{The Complete code}\label{the-complete-code}

You can find the source files in the \passthrough{\lstinline!tfc!}
directory. it can be compiled with \passthrough{\lstinline!./comp tfc!}.

\passthrough{\lstinline!tfc.c!}

\begin{lstlisting}[language=C, numbers=left]
#include <gtk/gtk.h>
#include <math.h>
#include <time.h>

float m_radius     = 0.42;
float m_line_width = 0.05;

static void
draw_clock (GtkDrawingArea *area, cairo_t *cr, int width, int height, gpointer user_data) {

    // Scale to unit square and translate (0, 0) to be (0.5, 0.5), i.e.
    // the center of the window
    cairo_scale(cr, width, height);
    cairo_translate(cr, 0.5, 0.5);

    // Set the line width and save the cairo drawing state.
    cairo_set_line_width(cr, m_line_width);
    cairo_save(cr);

    // Set the background to a slightly transparent green.
    cairo_set_source_rgba(cr, 0.337, 0.612, 0.117, 0.9);   // green
    cairo_paint(cr);

    // Resore back to precious drawing state and draw the circular path
    // representing the clockface. Save this state (including the path) so we
    // can reuse it.
    cairo_restore(cr);
    cairo_arc(cr, 0.0, 0.0, m_radius, 0.0, 2.0 * M_PI);
    cairo_save(cr);

    // Fill the clockface with white
    cairo_set_source_rgba(cr, 1.0, 1.0, 1.0, 0.8);
    cairo_fill_preserve(cr);
    // Restore the path, paint the outside of the clock face.
    cairo_restore(cr);
    cairo_stroke_preserve(cr);
    // Set the 'clip region' to the inside of the path (fill region).
    cairo_clip(cr);

    // Clock ticks
    for (int i = 0; i < 12; i++)
    {
        // Major tick size
        double inset = 0.05;

        // Save the graphics state, restore after drawing tick to maintain pen
        // size
        cairo_save(cr);
        cairo_set_line_cap(cr, CAIRO_LINE_CAP_ROUND);

        // Minor ticks are shorter, and narrower.
        if(i % 3 != 0)
        {
            inset *= 0.8;
            cairo_set_line_width(cr, 0.03);
        }

        // Draw tick mark
        cairo_move_to(
            cr,
            (m_radius - inset) * cos (i * M_PI / 6.0),
            (m_radius - inset) * sin (i * M_PI / 6.0));
        cairo_line_to(
            cr,
            m_radius * cos (i * M_PI / 6.0),
            m_radius * sin (i * M_PI / 6.0));
        cairo_stroke(cr);
        cairo_restore(cr); /* stack-pen-size */
    }

    // Draw the analog hands

    // Get the current Unix time, convert to the local time and break into time
    // structure to read various time parts.
    time_t rawtime;
    time(&rawtime);
    struct tm * timeinfo = localtime (&rawtime);

    // Calculate the angles of the hands of our clock
    double hours   = timeinfo->tm_hour * M_PI / 6.0;
    double minutes = timeinfo->tm_min * M_PI / 30.0;
    double seconds = timeinfo->tm_sec * M_PI / 30.0;

    // Save the graphics state
    cairo_save(cr);
    cairo_set_line_cap(cr, CAIRO_LINE_CAP_ROUND);

    cairo_save(cr);

    // Draw the seconds hand
    cairo_set_line_width(cr, m_line_width / 3.0);
    cairo_set_source_rgba(cr, 0.7, 0.7, 0.7, 0.8);   // gray
    cairo_move_to(cr, 0.0, 0.0);
    cairo_line_to(cr,
                  sin(seconds) * (m_radius * 0.9),
                  -cos(seconds) * (m_radius * 0.9));
    cairo_stroke(cr);
    cairo_restore(cr);

    // Draw the minutes hand
    cairo_set_source_rgba(cr, 0.117, 0.337, 0.612, 0.9);   // blue
    cairo_move_to(cr, 0, 0);
    cairo_line_to(cr,
                  sin(minutes + seconds / 60) * (m_radius * 0.8),
                  -cos(minutes + seconds / 60) * (m_radius * 0.8));
    cairo_stroke(cr);

    // draw the hours hand
    cairo_set_source_rgba(cr, 0.337, 0.612, 0.117, 0.9);   // green
    cairo_move_to(cr, 0.0, 0.0);
    cairo_line_to(cr,
                  sin(hours + minutes / 12.0) * (m_radius * 0.5),
                  -cos(hours + minutes / 12.0) * (m_radius * 0.5));
    cairo_stroke(cr);
    cairo_restore(cr);

    // Draw a little dot in the middle
    cairo_arc(cr, 0.0, 0.0, m_line_width / 3.0, 0.0, 2.0 * M_PI);
    cairo_fill(cr);
}


gboolean
time_handler(GtkWidget* widget) {
    gtk_widget_queue_draw(widget);

    return TRUE;
}


static void
app_activate (GApplication *app, gpointer user_data) {
    GtkWidget *win;
    GtkWidget *clock;
    GtkBuilder *build;

    build = gtk_builder_new_from_resource ("/com/github/ToshioCP/tfc/tfc.ui");
    win = GTK_WIDGET (gtk_builder_get_object (build, "win"));
    gtk_window_set_application (GTK_WINDOW (win), GTK_APPLICATION (app));

    clock = GTK_WIDGET (gtk_builder_get_object (build, "clock"));
    g_object_unref(build);

    gtk_drawing_area_set_draw_func(GTK_DRAWING_AREA (clock), draw_clock, NULL, NULL);
    g_timeout_add(1000, (GSourceFunc) time_handler, (gpointer) clock);
    gtk_window_present(GTK_WINDOW (win));

}

static void
app_open (GApplication *app, GFile **files, gint n_files, gchar *hint, gpointer user_data) {
    app_activate(app,user_data);
}

int
main (int argc, char **argv) {
    GtkApplication *app;
    int stat;

    app = gtk_application_new ("com.github.ToshioCP.tfc", G_APPLICATION_HANDLES_OPEN);
    g_signal_connect (app, "activate", G_CALLBACK (app_activate), NULL);
    g_signal_connect (app, "open", G_CALLBACK (app_open), NULL);
    stat = g_application_run (G_APPLICATION (app), argc, argv);
    g_object_unref (app);
    return stat;
}
\end{lstlisting}

\passthrough{\lstinline!tfc.ui!}

\begin{lstlisting}[language=XML, numbers=left]
<?xml version="1.0" encoding="UTF-8"?>
<interface>
  <object class="GtkApplicationWindow" id="win">
    <property name="title">Clock</property>
    <property name="default-width">200</property>
    <property name="default-height">200</property>
    <child>
      <object class="GtkDrawingArea" id="clock">
        <property name="hexpand">TRUE</property>
        <property name="vexpand">TRUE</property>
      </object>
    </child>
  </object>
</interface>
\end{lstlisting}

\passthrough{\lstinline!tfc.gresource.xml!}

\begin{lstlisting}[language=XML, numbers=left]
<?xml version="1.0" encoding="UTF-8"?>
<gresources>
  <gresource prefix="/com/github/ToshioCP/tfc">
    <file>tfc.ui</file>
  </gresource>
</gresources>
\end{lstlisting}

\passthrough{\lstinline!comp!}

\begin{lstlisting}[numbers=left]
glib-compile-resources $1.gresource.xml --target=$1.gresource.c --generate-source
gcc `pkg-config --cflags gtk4` $1.gresource.c $1.c `pkg-config --libs gtk4` -lm
\end{lstlisting}
