\section{Build system}\label{build-system}

\subsection{Managing big source files}\label{managing-big-source-files}

We've compiled a small editor so far. The program is also small and not
complicated yet. But if it grows bigger, it will be difficult to
maintain. So, we should do the followings now.

\begin{itemize}
\tightlist
\item
  We've had only one C source file and put everything in it. We need to
  divide it and sort them out.
\item
  There are two compilers, \passthrough{\lstinline!gcc!} and
  \passthrough{\lstinline!glib-compile-resources!}. We should control
  them by one building tool.
\end{itemize}

\subsection{Divide a C source file into two
parts.}\label{divide-a-c-source-file-into-two-parts.}

When you divide C source file into several parts, each file should
contain one thing. For example, our source has two things, the
definition of TfeTextView and functions related to GtkApplication and
GtkApplicationWindow. It is a good idea to separate them into two files,
\passthrough{\lstinline!tfetextview.c!} and
\passthrough{\lstinline!tfe.c!}.

\begin{itemize}
\tightlist
\item
  \passthrough{\lstinline!tfetextview.c!} includes the definition and
  functions of TfeTextView.
\item
  \passthrough{\lstinline!tfe.c!} includes functions like
  \passthrough{\lstinline!main!},
  \passthrough{\lstinline!app\_activate!},
  \passthrough{\lstinline!app\_open!} and so on, which relate to
  GtkApplication and GtkApplicationWindow
\end{itemize}

Now we have three source files, \passthrough{\lstinline!tfetextview.c!},
\passthrough{\lstinline!tfe.c!} and \passthrough{\lstinline!tfe3.ui!}.
The \passthrough{\lstinline!3!} of \passthrough{\lstinline!tfe3.ui!} is
like a version number. Managing version with filenames is one possible
idea but it also has a problem. You need to rewrite filename in each
version and it affects to contents of source files that refer to
filenames. So, we should take \passthrough{\lstinline!3!} away from the
filename.

In \passthrough{\lstinline!tfe.c!} the function
\passthrough{\lstinline!tfe\_text\_view\_new!} is invoked to create a
TfeTextView instance. But it is defined in
\passthrough{\lstinline!tfetextview.c!}, not
\passthrough{\lstinline!tfe.c!}. The lack of the declaration (not
definition) of \passthrough{\lstinline!tfe\_text\_view\_new!} makes
error when \passthrough{\lstinline!tfe.c!} is compiled. The declaration
is necessary in \passthrough{\lstinline!tfe.c!}. Those public
information is usually written in header files. It has
\passthrough{\lstinline!.h!} suffix like
\passthrough{\lstinline!tfetextview.h!}. And header files are included
by C source files. For example, \passthrough{\lstinline!tfetextview.h!}
is included by \passthrough{\lstinline!tfe.c!}.

The source files are shown below.

\passthrough{\lstinline!tfetextview.h!}

\begin{lstlisting}[language=C, numbers=left]
#include <gtk/gtk.h>

#define TFE_TYPE_TEXT_VIEW tfe_text_view_get_type ()
G_DECLARE_FINAL_TYPE (TfeTextView, tfe_text_view, TFE, TEXT_VIEW, GtkTextView)

void
tfe_text_view_set_file (TfeTextView *tv, GFile *f);

GFile *
tfe_text_view_get_file (TfeTextView *tv);

GtkWidget *
tfe_text_view_new (void);
\end{lstlisting}

\passthrough{\lstinline!tfetextview.c!}

\begin{lstlisting}[language=C, numbers=left]
#include <gtk/gtk.h>
#include "tfetextview.h"

struct _TfeTextView
{
  GtkTextView parent;
  GFile *file;
};

G_DEFINE_TYPE (TfeTextView, tfe_text_view, GTK_TYPE_TEXT_VIEW);

static void
tfe_text_view_init (TfeTextView *tv) {
}

static void
tfe_text_view_class_init (TfeTextViewClass *class) {
}

void
tfe_text_view_set_file (TfeTextView *tv, GFile *f) {
  tv->file = f;
}

GFile *
tfe_text_view_get_file (TfeTextView *tv) {
  return tv->file;
}

GtkWidget *
tfe_text_view_new (void) {
  return GTK_WIDGET (g_object_new (TFE_TYPE_TEXT_VIEW, NULL));
}
\end{lstlisting}

\passthrough{\lstinline!tfe.c!}

\begin{lstlisting}[language=C, numbers=left]
#include <gtk/gtk.h>
#include "tfetextview.h"

static void
app_activate (GApplication *app) {
  g_print ("You need a filename argument.\n");
}

static void
app_open (GApplication *app, GFile ** files, gint n_files, gchar *hint) {
  GtkWidget *win;
  GtkWidget *nb;
  GtkWidget *lab;
  GtkNotebookPage *nbp;
  GtkWidget *scr;
  GtkWidget *tv;
  GtkTextBuffer *tb;
  char *contents;
  gsize length;
  char *filename;
  int i;
  GError *err = NULL;
  GtkBuilder *build;

  build = gtk_builder_new_from_resource ("/com/github/ToshioCP/tfe/tfe.ui");
  win = GTK_WIDGET (gtk_builder_get_object (build, "win"));
  gtk_window_set_application (GTK_WINDOW (win), GTK_APPLICATION (app));
  nb = GTK_WIDGET (gtk_builder_get_object (build, "nb"));
  g_object_unref (build);
  for (i = 0; i < n_files; i++) {
    if (g_file_load_contents (files[i], NULL, &contents, &length, NULL, &err)) {
      scr = gtk_scrolled_window_new ();
      tv = tfe_text_view_new ();
      tb = gtk_text_view_get_buffer (GTK_TEXT_VIEW (tv));
      gtk_text_view_set_wrap_mode (GTK_TEXT_VIEW (tv), GTK_WRAP_WORD_CHAR);
      gtk_scrolled_window_set_child (GTK_SCROLLED_WINDOW (scr), tv);

      tfe_text_view_set_file (TFE_TEXT_VIEW (tv),  g_file_dup (files[i]));
      gtk_text_buffer_set_text (tb, contents, length);
      g_free (contents);
      filename = g_file_get_basename (files[i]);
      lab = gtk_label_new (filename);
      gtk_notebook_append_page (GTK_NOTEBOOK (nb), scr, lab);
      nbp = gtk_notebook_get_page (GTK_NOTEBOOK (nb), scr);
      g_object_set (nbp, "tab-expand", TRUE, NULL);
      g_free (filename);
    } else {
      g_printerr ("%s.\n", err->message);
      g_clear_error (&err);
    }
  }
  if (gtk_notebook_get_n_pages (GTK_NOTEBOOK (nb)) > 0) {
    gtk_window_present (GTK_WINDOW (win));
  } else
    gtk_window_destroy (GTK_WINDOW (win));
}

int
main (int argc, char **argv) {
  GtkApplication *app;
  int stat;

  app = gtk_application_new ("com.github.ToshioCP.tfe", G_APPLICATION_HANDLES_OPEN);
  g_signal_connect (app, "activate", G_CALLBACK (app_activate), NULL);
  g_signal_connect (app, "open", G_CALLBACK (app_open), NULL);
  stat =g_application_run (G_APPLICATION (app), argc, argv);
  g_object_unref (app);
  return stat;
}
\end{lstlisting}

The ui file \passthrough{\lstinline!tfe.ui!} is the same as
\passthrough{\lstinline!tfe3.ui!} in the previous section.

\passthrough{\lstinline!tfe.gresource.xml!}

\begin{lstlisting}[language=XML, numbers=left]
<?xml version="1.0" encoding="UTF-8"?>
<gresources>
  <gresource prefix="/com/github/ToshioCP/tfe">
    <file>tfe.ui</file>
  </gresource>
</gresources>
\end{lstlisting}

Dividing a file makes it easy to maintain. But now we face a new
problem. The building step increases.

\begin{itemize}
\tightlist
\item
  Compiling the ui file \passthrough{\lstinline!tfe.ui!} into
  \passthrough{\lstinline!resources.c!}.
\item
  Compiling \passthrough{\lstinline!tfe.c!} into
  \passthrough{\lstinline!tfe.o!} (object file).
\item
  Compiling \passthrough{\lstinline!tfetextview.c!} into
  \passthrough{\lstinline!tfetextview.o!}.
\item
  Compiling \passthrough{\lstinline!resources.c!} into
  \passthrough{\lstinline!resources.o!}.
\item
  Linking all the object files into application
  \passthrough{\lstinline!tfe!}.
\end{itemize}

Build tools manage the steps.

\subsection{Meson and Ninja}\label{meson-and-ninja}

I'll explain Meson and Ninja build tools.

Other possible tools are Make and Autotools. They are traditional tools
but slower than Ninja. So, many developers use Meson and Ninja lately.
For example, GTK 4 uses them.

You need to create \passthrough{\lstinline!meson.build!} file first.

\begin{lstlisting}[numbers=left]
project('tfe', 'c')

gtkdep = dependency('gtk4')

gnome=import('gnome')
resources = gnome.compile_resources('resources','tfe.gresource.xml')

sourcefiles=files('tfe.c', 'tfetextview.c')

executable('tfe', sourcefiles, resources, dependencies: gtkdep, install: false)
\end{lstlisting}

\begin{itemize}
\tightlist
\item
  1: The function \passthrough{\lstinline!project!} defines things about
  the project. The first argument is the name of the project and the
  second is the programming language.
\item
  3: The function \passthrough{\lstinline!dependency!} defines a
  dependency that is taken by \passthrough{\lstinline!pkg-config!}. We
  put \passthrough{\lstinline!gtk4!} as an argument.
\item
  5: The function \passthrough{\lstinline!import!} imports a module. In
  line 5, the gnome module is imported and assigned to the variable
  \passthrough{\lstinline!gnome!}. The gnome module provides helper
  tools to build GTK programs.
\item
  6: The method \passthrough{\lstinline!.compile\_resources!} is of the
  gnome module and compiles files to resources under the instruction of
  xml file. In line 6, the resource filename is
  \passthrough{\lstinline!resources!}, which means
  \passthrough{\lstinline!resources.c!} and
  \passthrough{\lstinline!resources.h!}, and xml file is
  \passthrough{\lstinline!tfe.gresource.xml!}. This method generates C
  source file by default.
\item
  8: Defines source files.
\item
  10: Executable function generates a target file by compiling source
  files. The first argument is the filename of the target. The following
  arguments are source files. The last two arguments have keys and
  values. For example, the fourth argument has a key
  \passthrough{\lstinline!dependencies!} , a delimiter
  (\passthrough{\lstinline!:!}) and a value
  \passthrough{\lstinline!gtkdep!}. This type of parameter is called
  \emph{keyword parameter} or \emph{kwargs}. The value
  \passthrough{\lstinline!gtkdep!} is defined in line 3. The last
  argument tells that this project doesn't install the executable file.
  So it is just compiled in the build directory.
\end{itemize}

Now run meson and ninja.

\begin{lstlisting}
$ meson setup _build
$ ninja -C _build
\end{lstlisting}

meson has two arguments.

\begin{itemize}
\tightlist
\item
  setup: The first argument is a command of meson. Setup is the default,
  so you can leave it out. But it is recommended to write it explicitly
  since version 0.64.0.
\item
  The second argument is the name of the build directory.
\end{itemize}

Then, the executable file \passthrough{\lstinline!tfe!} is generated
under the directory \passthrough{\lstinline!\_build!}.

\begin{lstlisting}
$ _build/tfe tfe.c tfetextview.c
\end{lstlisting}

A window appears. It includes a notebook with two pages. One is
\passthrough{\lstinline!tfe.c!} and the other is
\passthrough{\lstinline!tfetextview.c!}.

For further information, see \href{https://mesonbuild.com/}{The Meson
Build system}.
