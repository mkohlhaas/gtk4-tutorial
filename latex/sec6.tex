\section{Strings and memory
management}\label{strings-and-memory-management}

GtkTextView and GtkTextBuffer have functions that have string parameters
or return a string. The knowledge of strings and memory management is
useful to understand how to use these functions.

\subsection{String and memory}\label{string-and-memory}

A String is an array of characters that is terminated with
`\textbackslash0'. String is not a C type such as char, int, float or
double, but a character array. It behaves like a string in other
languages. So, the pointer is often called `a string'.

The following is a sample program.

\begin{lstlisting}[language=C]
char a[10], *b;

a[0] = 'H';
a[1] = 'e';
a[2] = 'l';
a[3] = 'l';
a[4] = 'o';
a[5] = '\0';

b = a;
/* *b is 'H' */
/* *(++b) is 'e' */
\end{lstlisting}

An array \passthrough{\lstinline!a!} is defined as a
\passthrough{\lstinline!char!} type array and its size is ten. The first
five elements are `H', `e', `l', `l', `o'. They are character codes. For
example, `H' is the same as 0x48 or 72. The sixth element is
`\textbackslash0', which is the same as zero, and indicates that the
sequence of the data ends there. The array represents the string
``Hello''.

The size of the array is 10, so four bytes aren't used. But it's OK.
They are just ignored. (If the variable \passthrough{\lstinline!a!} is
defined out of functions or its class is static, the four bytes are
assigned with zero. Otherwise, that is to say, the class is auto or
register, they are undefined.)

The variable \passthrough{\lstinline!b!} is a pointer to a character. It
is assigned with \passthrough{\lstinline!a!}, so
\passthrough{\lstinline!b!} points the first element of
\passthrough{\lstinline!a!} (character `H'). The array
\passthrough{\lstinline!a!} is immutable. So
\passthrough{\lstinline!a=a+1!} causes syntax error.

On the other hand, \passthrough{\lstinline!b!} is a pointer type
variable, which is mutable. So, \passthrough{\lstinline!++b!}, which
increases \passthrough{\lstinline!b!} by one, is allowed.

If a pointer is NULL, it points nothing. So, the pointer is not a
string. It is different from empty string. Empty string is a pointer
points `\textbackslash0'.

There are four cases:

\begin{itemize}
\tightlist
\item
  The string is read only
\item
  The string is in static memory area
\item
  The string is in stack
\item
  The string is in memory allocated from the heap area
\end{itemize}

\subsection{Read only string}\label{read-only-string}

A string literal is surrounded by double quotes like this:

\begin{lstlisting}[language=C]
char *s;
s = "Hello"
\end{lstlisting}

``Hello'' is a string literal, and is read only. So, the following
program is illegal.

\begin{lstlisting}[language=C]
*(s+1) = 'a';
\end{lstlisting}

The result is undefined. Probably a bad thing will happen, for example,
a segmentation fault.

NOTE: The memory of the literal string is allocated when the program is
compiled. It is possible to see the literal strings with
\passthrough{\lstinline!strings!} command.

\begin{lstlisting}
$ strings src/tvf/a.out
/lib64/ld-linux-x86-64.so.2
cN<5
... ... ...
... ... ...
Once upon a time, there was an old man who was called Taketori-no-Okina. It is a japanese word that means a man whose work is making bamboo baskets.
One day, he went into a hill and found a shining bamboo. "What a mysterious bamboo it is!," he said. He cut it, then there was a small cute baby girl in it. The girl was shining faintly. He thought this baby girl is a gift from Heaven and took her home.
His wife was surprized at his story. They were very happy because they had no children. 
... ... ...
... ... ...
\end{lstlisting}

It tells us that literal strings are embedded in program binary codes.

\subsection{Strings defined as arrays}\label{strings-defined-as-arrays}

If a string is defined as an array, it's stored in static memory area or
stack. It depends on the class of the array. If the array's class is
\passthrough{\lstinline!static!}, then it's placed in static memory
area. The allocated memory lives for the life of the program. This area
is writable.

If the array's class is \passthrough{\lstinline!auto!}, it's placed in
stack. If the array is defined inside a function, its default class is
\passthrough{\lstinline!auto!}. The stack area will disappear when the
function returns to the caller. Arrays defined on the stack are
writable.

\begin{lstlisting}[language=C]
static char a[] = {'H', 'e', 'l', 'l', 'o', '\0'};

void
print_strings (void) {
  char b[] = "Hello";

  a[1] = 'a'; /* Because the array is static, it's writable. */
  b[1] = 'a'; /* Because the array is auto, it's writable. */

  printf ("%s\n", a); /* Hallo */
  printf ("%s\n", b); /* Hallo */
}
\end{lstlisting}

The array \passthrough{\lstinline!a!} is defined out of functions. It is
placed in the static memory area even if the
\passthrough{\lstinline!static!} class is left out. The compiler
calculates the number of the elements (six) and allocates six bytes in
the static memory area. Then, it copies ``Hello'' literal string data to
the memory.

The array \passthrough{\lstinline!b!} is defined inside the function, so
its class is \passthrough{\lstinline!auto!}. The compiler calculates the
number of the elements in the string literal. It is six because it has
`\textbackslash0' terminator. The compiler allocates six bytes in the
stack and copies ``Hello'' litaral string to the stack memory.

Both \passthrough{\lstinline!a!} and \passthrough{\lstinline!b!} are
writable.

The memory is allocated and freed by the program automatically so you
don't need to allocate or free. The array \passthrough{\lstinline!a!} is
alive during the program's life time. The array
\passthrough{\lstinline!b!} is alive when the function is called until
the function returns to the caller.

\subsection{Strings in the heap area}\label{strings-in-the-heap-area}

You can get, use and release memory from the heap area. The standard C
library provides \passthrough{\lstinline!malloc!} to get memory and
\passthrough{\lstinline!free!} to put back memory. GLib provides the
functions \passthrough{\lstinline!g\_new!} and
\passthrough{\lstinline!g\_free!}. They are similar to
\passthrough{\lstinline!malloc!} and \passthrough{\lstinline!free!}.

\begin{lstlisting}[language=C]
g_new (struct_type, n_struct)
\end{lstlisting}

\passthrough{\lstinline!g\_new!} is a macro to allocate memory for an
array.

\begin{itemize}
\tightlist
\item
  \passthrough{\lstinline!struct\_type!} is the type of the element of
  the array.
\item
  \passthrough{\lstinline!n\_struct!} is the size of the array.
\item
  The return value is a pointer to the array. Its type is a pointer to
  \passthrough{\lstinline!struct\_type!}.
\end{itemize}

For example,

\begin{lstlisting}[language=C]
char *s;
s = g_new (char, 10);
/* s points an array of char. The size of the array is 10. */

struct tuple {int x, y;} *t;
t = g_new (struct tuple, 5);
/* t points an array of struct tuple. */
/* The size of the array is 5. */
\end{lstlisting}

\passthrough{\lstinline!g\_free!} frees memory.

\begin{lstlisting}[language=C]
void
g_free (gpointer mem);
\end{lstlisting}

If \passthrough{\lstinline!mem!} is NULL,
\passthrough{\lstinline!g\_free!} does nothing.
\passthrough{\lstinline!gpointer!} is a type of general pointer. It is
the same as \passthrough{\lstinline!void *!}. This pointer can be casted
to any pointer type. Conversely, any pointer type can be casted to
\passthrough{\lstinline!gpointer!}.

\begin{lstlisting}[language=C]
g_free (s);
/* Frees the memory allocated to s. */

g_free (t);
/* Frees the memory allocated to t. */
\end{lstlisting}

If the argument doesn't point allocated memory it will cause an error,
specifically, a segmentation fault.

Some GLib functions allocate memory. For example,
\passthrough{\lstinline!g\_strdup!} allocates memory and copies a string
given as an argument.

\begin{lstlisting}[language=C]
char *s;
s = g_strdup ("Hello");
g_free (s);
\end{lstlisting}

The string literal ``Hello'' has 6 bytes because the string has
`\textbackslash0' at the end. \passthrough{\lstinline!g\_strdup!} gets 6
bytes from the heap area and copies the string to the memory.
\passthrough{\lstinline!s!} is assigned the start address of the memory.
\passthrough{\lstinline!g\_free!} returns the memory to the heap area.

\passthrough{\lstinline!g\_strdup!} is described in
\href{https://docs.gtk.org/glib/func.strdup.html}{GLib API Reference}.
The following is extracted from the reference.

\begin{quote}
The returned string should be freed with
\passthrough{\lstinline!g\_free()!} when no longer needed.
\end{quote}

If you forget to free the allocated memory it will remain until the
program ends. Repeated allocation and no freeing cause memory leak. It
is a bug and may bring a serious problem.

\subsection{const qualifier}\label{const-qualifier}

A \passthrough{\lstinline!const!} qualified variable can be assigned to
initialize it. Once it is initialized, it is never allowed to change or
free.

\begin{lstlisting}[language=C]
const int x = 10; /* initialization is OK. */

x = 20; /* This is illegal because x is qualified with const */
\end{lstlisting}

If a function returns \passthrough{\lstinline!const char*!} type, the
string can't be changed or freed. If a function has a
\passthrough{\lstinline!const char *!} type parameter, it ensures that
the parameter is not changed in the function.

\begin{lstlisting}[language=C]
// You never change or free the returned string.
const char*
gtk_label_get_text (
  GtkLabel* self
)

// Str keeps itself during the function runs
void
gtk_label_set_text (
  GtkLabel* self,
  const char* str
)
\end{lstlisting}
