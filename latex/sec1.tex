\section{Prerequisite and License}\label{prerequisite-and-license}

\subsection{Prerequisite}\label{prerequisite}

\subsubsection{GTK 4 on a Linux OS}\label{gtk-4-on-a-linux-os}

This tutorial describes GTK 4 libraries. It is originally used on Linux
with C compiler, but now it is used more widely, on Windows and MacOS,
with Vala, Python, Ruby and so on. However, this tutorial describes only
\emph{C programs on Linux}.

If you want to try the examples in the tutorial, you need:

\begin{itemize}
\tightlist
\item
  PC with Linux distribution like Ubuntu or Debian.
\item
  GCC.
\item
  GTK 4 (version 4.10.1 or later).
\end{itemize}

The stable version of GTK is 4.10.1 at present (24/April/2023). The
version 4.10 adds some new classes and functions and makes some classes
and functions deprecated. Some example programs in this tutorial don't
work on the older version.

\subsubsection{Ruby and rake for making the
document}\label{ruby-and-rake-for-making-the-document}

This repository includes Ruby programs. They are used to make GFM
(GitHub Flavoured Markdown) files, HTML files, Latex files and a PDF
file.

You need:

\begin{itemize}
\tightlist
\item
  Linux.
\item
  Ruby programming language. There are two ways to install. One is
  installing the distribution's package. The other is using rbenv and
  ruby-build. If you want to use the latest version of ruby, use rbenv.
\item
  Rake.
\end{itemize}

\subsection{License}\label{license}

Copyright (C) 2020,2023 ToshioCP (Toshio Sekiya)

GTK4-tutorial repository contains tutorial documents and programs such
as converters, generators and controllers. All of them make up the
`GTK4-tutorial' package. This package is simply called `GTK4-tutorial'
in the following description.

GTK4-tutorial is free; you can redistribute it and/or modify it under
the terms of the following licenses.

\begin{itemize}
\tightlist
\item
  The license of documents in GTK4-tutorial is the GNU Free
  Documentation License as published by the Free Software Foundation;
  either version 1.3 of the License or, at your opinion, any later
  version. The documents are Markdown, HTML and image files. If you
  generate a PDF file by running \passthrough{\lstinline!rake pdf!}, it
  is also included by the documents.
\item
  The license of programs in GTK4-tutorial is the GNU General Public
  License as published by the Free Software Foundation; either version 3
  of the License or, at your option, any later version. The programs are
  written in C, Ruby and other languages.
\end{itemize}

GTK4-tutorial is distributed in the hope that it will be useful, but
WITHOUT ANY WARRANTY; without even the implied warranty of
MERCHANTABILITY or FITNESS FOR A PARTICULAR PURPOSE. See the GNU License
web pages for more details.

\begin{itemize}
\tightlist
\item
  \href{https://www.gnu.org/licenses/fdl-1.3.html}{GNU Free
  Documentation License}
\item
  \href{https://www.gnu.org/licenses/gpl-3.0.html}{GNU General Public
  License}
\end{itemize}

The licenses above is effective since 15/April/2023. Before that, GPL
covered all the contents of the GTK4-tutorial. But GFDL1.3 is more
appropriate for documents so the license was changed. The license above
is the only effective license since 15/April/2023.
