\paragraph{Contents of this
Repository}\label{contents-of-this-repository}

This tutorial illustrates how to write C programs with the GTK 4
library. It focuses on beginners so the contents are limited to the
basics. The table of contents is at the end of this abstract.

\begin{itemize}
\tightlist
\item
  Section 3 to 23 describes the basics, with the example of a simple
  editor \passthrough{\lstinline!tfe!} (Text File Editor).
\item
  Section 24 to 27 describes GtkDrawingArea.
\item
  Section 28 describes Drag and Drop.
\item
  Section 29 to 33 describes the list model and the list view
  (GtkListView, GtkGridView and GtkColumnView). It also describes
  GtkExpression.
\end{itemize}

The latest version of the tutorial is located at
\href{https://github.com/ToshioCP/Gtk4-tutorial}{GTK4-tutorial GitHub
repository}. You can read it directly without download.

There's a \href{https://toshiocp.github.io/Gtk4-tutorial/}{GitHub Page}
which is the HTML version of the tutorial.

\paragraph{GTK 4 Documentation}\label{gtk-4-documentation}

Please refer to \href{https://docs.gtk.org/gtk4/index.html}{GTK 4 API
Reference} and \href{https://developer.gnome.org/}{GNOME Developer
Documentation Website} for further information.

These websites were opened in August of 2021. The old documents are
located at \href{https://developer-old.gnome.org/gtk4/stable/}{GTK
Reference Manual} and \href{https://developer-old.gnome.org/}{GNOME
Developer Center}.

If you want to know about GObject and the type system, please refer to
\href{https://github.com/ToshioCP/Gobject-tutorial}{GObject tutorial}.
GObject is the base of GTK 4, so it is important for developers to
understand it as well as GTK 4.

\paragraph{Contribution}\label{contribution}

This tutorial is still under development and unstable. Even though the
codes of the examples have been tested on GTK 4 (version 4.10.1), bugs
may still exist. If you find any bugs, errors or mistakes in the
tutorial and C examples, please let me know. You can post it to
\href{https://github.com/ToshioCP/Gtk4-tutorial/issues}{GitHub issues}.
You can also post updated files to
\href{https://github.com/ToshioCP/Gtk4-tutorial/pulls}{pull request}.
One thing you need to be careful is to correct the source files, which
are under the `src' directory. Don't modify the files under
\passthrough{\lstinline!gfm!} or \passthrough{\lstinline!html!}
directories. After modifying some source files , run
\passthrough{\lstinline!rake!} to create GFM (GitHub Flavoured Markdown)
files and run \passthrough{\lstinline!rake html!} to create HTML files.

If you have a question, feel free to post it to
\passthrough{\lstinline!issue!}. All questions are helpful and will make
this tutorial get better.

\paragraph{How to get Gtk 4 tutorial with HTML or PDF
format}\label{how-to-get-gtk-4-tutorial-with-html-or-pdf-format}

If you want to get HTML or PDF format tutorial, make them with
\passthrough{\lstinline!rake!} command, which is a ruby version of make.
Type \passthrough{\lstinline!rake html!} for HTML. Type
\passthrough{\lstinline!rake pdf!} for PDF. An appendix ``How to build
GTK 4 Tutorial'' describes how to make them.

\paragraph{License}\label{license}

The license of this repository is written in Section1. In short,

\begin{itemize}
\tightlist
\item
  GFDL1.3 for documents
\item
  GPL3 for programs
\end{itemize}
