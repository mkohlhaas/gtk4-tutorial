\section{Signals}\label{signals}

\subsection{Signals}\label{signals-1}

Each object is encapsulated in Gtk programming. And it is not
recommended to use global variables because they are prone to make the
program complicated. So, we need something to communicate between
objects. There are two ways to do so.

\begin{itemize}
\tightlist
\item
  Instance methods: Instance methods are functions on instances. For
  example,
  \passthrough{\lstinline!tb = gtk\_text\_view\_get\_buffer (tv)!} is an
  instance method on the instance \passthrough{\lstinline!tv!}. The
  caller requests \passthrough{\lstinline!tv!} to give
  \passthrough{\lstinline!tb!}, which is a GtkTextBuffer instance
  connected to \passthrough{\lstinline!tv!}.
\item
  Signals: For example, \passthrough{\lstinline!activate!} signal on
  GApplication object. When the application is activated, the signal is
  emitted. Then the handler, which has been connected to the signal, is
  invoked.
\end{itemize}

The caller of methods or signals are usually out of the object. One of
the difference between these two is that the object is active or
passive. In methods, objects passively respond to the caller. In
signals, objects actively send signals to handlers.

GObject signals are registered, connected and emitted.

\begin{enumerate}
\def\labelenumi{\arabic{enumi}.}
\tightlist
\item
  Signals are registered in the class. The registration is done usually
  when the class is initialized. Signals can have a default handler,
  which is sometimes called ``object method handler''. It is not a user
  handler connected by \passthrough{\lstinline!g\_signal\_connect!}
  family functions. A default handler is always called on any instance
  of the class.
\item
  Signals are connected to handlers by the macro
  \passthrough{\lstinline!g\_signal\_connect!} or its family functions.
  The connection is usually done out of the object. One important thing
  is that signals are connected on a certain instance. Suppose there
  exist two GtkButton instances A, B and a function C. Even if you
  connected the ``clicked'' signal on A to C, B and C are \emph{not}
  connected.
\item
  When Signals are emitted, the connected handlers are invoked. Signals
  are emitted on the instance of the class.
\end{enumerate}

\subsection{Signal registration}\label{signal-registration}

In TfeTextView, two signals are registered.

\begin{itemize}
\tightlist
\item
  ``change-file'' signal. This signal is emitted when
  \passthrough{\lstinline!tv->file!} is changed.
\item
  ``open-response'' signal. The function
  \passthrough{\lstinline!tfe\_text\_view\_open!} doesn't return the
  status because it can't get the status from the file chooser dialog.
  (Instead, the call back function gets the status.) This signal is
  emitted instead of the return value of the function.
\end{itemize}

A static variable or array is used to store signal ID.

\begin{lstlisting}[language=C]
enum {
  CHANGE_FILE,
  OPEN_RESPONSE,
  NUMBER_OF_SIGNALS
};

static guint tfe_text_view_signals[NUMBER_OF_SIGNALS];
\end{lstlisting}

Signals are registered in the class initialization function.

\begin{lstlisting}[language=C, numbers=left]
static void
tfe_text_view_class_init (TfeTextViewClass *class) {
  GObjectClass *object_class = G_OBJECT_CLASS (class);

  object_class->dispose = tfe_text_view_dispose;
  tfe_text_view_signals[CHANGE_FILE] = g_signal_new ("change-file",
                                 G_TYPE_FROM_CLASS (class),
                                 G_SIGNAL_RUN_LAST | G_SIGNAL_NO_RECURSE | G_SIGNAL_NO_HOOKS,
                                 0 /* class offset */,
                                 NULL /* accumulator */,
                                 NULL /* accumulator data */,
                                 NULL /* C marshaller */,
                                 G_TYPE_NONE /* return_type */,
                                 0     /* n_params */
                                 );
  tfe_text_view_signals[OPEN_RESPONSE] = g_signal_new ("open-response",
                                 G_TYPE_FROM_CLASS (class),
                                 G_SIGNAL_RUN_LAST | G_SIGNAL_NO_RECURSE | G_SIGNAL_NO_HOOKS,
                                 0 /* class offset */,
                                 NULL /* accumulator */,
                                 NULL /* accumulator data */,
                                 NULL /* C marshaller */,
                                 G_TYPE_NONE /* return_type */,
                                 1     /* n_params */,
                                 G_TYPE_INT
                                 );
}
\end{lstlisting}

\begin{itemize}
\tightlist
\item
  6-15: Registers ``change-file'' signal.
  \passthrough{\lstinline!g\_signal\_new!} function is used. The signal
  ``change-file'' has no default handler (object method handler) so the
  offset (the line 9) is set to zero. You usually don't need a default
  handler. If you need it, use
  \passthrough{\lstinline!g\_signal\_new\_class\_handler!} function
  instead of \passthrough{\lstinline!g\_signal\_new!}. See
  \href{https://docs.gtk.org/gobject/func.signal_new_class_handler.html}{GObject
  API Reference} for further information.
\item
  The return value of \passthrough{\lstinline!g\_signal\_new!} is the
  signal id. The type of signal id is guint, which is the same as
  unsigned int. It is used in the function
  \passthrough{\lstinline!g\_signal\_emit!}.
\item
  16-26: Registers ``open-response'' signal. This signal has a
  parameter.
\item
  24: Number of the parameters. ``open-response'' signal has one
  parameter.
\item
  25: The type of the parameter. \passthrough{\lstinline!G\_TYPE\_INT!}
  is a type of integer. Such fundamental types are described in
  \href{https://developer-old.gnome.org/gobject/stable/gobject-Type-Information.html}{GObject
  reference manual}.
\end{itemize}

The handlers are declared as follows.

\begin{lstlisting}[language=C]
/* "change-file" signal handler */
void
user_function (TfeTextView *tv,
               gpointer user_data)

/* "open-response" signal handler */
void
user_function (TfeTextView *tv,
               TfeTextViewOpenResponseType response-id,
               gpointer user_data)
\end{lstlisting}

\begin{itemize}
\tightlist
\item
  The signal ``change-file'' doesn't have parameter, so the handler's
  arguments are a TfeTextView instance and a user data.
\item
  The signal ``open-response'' signal has one parameter and its
  arguments are a TfeTextView instance, the signal's parameter and user
  data.
\item
  The variable \passthrough{\lstinline!tv!} points the instance on which
  the signal is emitted.
\item
  The last argument \passthrough{\lstinline!user\_data!} comes from the
  fourth argument of \passthrough{\lstinline!g\_signal\_connect!}.
\item
  The \passthrough{\lstinline!parameter!}
  (\passthrough{\lstinline!response-id!}) comes from the fourth argument
  of \passthrough{\lstinline!g\_signal\_emit!}.
\end{itemize}

The values of the type
\passthrough{\lstinline!TfeTextViewOpenResponseType!} are defined in
\passthrough{\lstinline!tfetextview.h!}.

\begin{lstlisting}[language=C]
/* "open-response" signal response */
enum TfeTextViewOpenResponseType
{
  TFE_OPEN_RESPONSE_SUCCESS,
  TFE_OPEN_RESPONSE_CANCEL,
  TFE_OPEN_RESPONSE_ERROR
};
\end{lstlisting}

\begin{itemize}
\tightlist
\item
  The parameter is set to
  \passthrough{\lstinline!TFE\_OPEN\_RESPONSE\_SUCCESS!} when
  \passthrough{\lstinline!tfe\_text\_view\_open!} has successfully
  opened a file and read it.
\item
  The parameter is set to
  \passthrough{\lstinline!TFE\_OPEN\_RESPONSE\_CANCEL!} when the user
  has canceled.
\item
  The parameter is set to
  \passthrough{\lstinline!TFE\_OPEN\_RESPONSE\_ERROR!} when an error has
  occurred.
\end{itemize}

\subsection{Signal connection}\label{signal-connection}

A signal and a handler are connected by the function macro
\passthrough{\lstinline!g\_signal\_connect!}. There are some similar
function macros like
\passthrough{\lstinline!g\_signal\_connect\_after!},
\passthrough{\lstinline!g\_signal\_connect\_swapped!} and so on.
However, \passthrough{\lstinline!g\_signal\_connect!} is used most
often. The signals ``change-file'' and ``open-response'' are connected
to their callback functions out of the TfeTextView object. Those
callback functions are defined by users.

For example, callback functions are defined in
\passthrough{\lstinline!src/tfe6/tfewindow.c!} and their names are
\passthrough{\lstinline!file\_changed\_cb!} and
\passthrough{\lstinline!open\_response\_cb!}. They will be explained
later.

\begin{lstlisting}[language=C]
g_signal_connect (GTK_TEXT_VIEW (tv), "change-file", G_CALLBACK (file_changed_cb), nb);

g_signal_connect (TFE_TEXT_VIEW (tv), "open-response", G_CALLBACK (open_response_cb), nb);
\end{lstlisting}

\subsection{Signal emission}\label{signal-emission}

A signal is emitted on the instance. A function
\passthrough{\lstinline!g\_signal\_emit!} is used to emit the signal.
The following lines are extracted from
\passthrough{\lstinline!src/tfetextview/tfetextview.c!}. Each line comes
from a different line.

\begin{lstlisting}[language=C]
g_signal_emit (tv, tfe_text_view_signals[CHANGE_FILE], 0);
g_signal_emit (tv, tfe_text_view_signals[OPEN_RESPONSE], 0, TFE_OPEN_RESPONSE_SUCCESS);
g_signal_emit (tv, tfe_text_view_signals[OPEN_RESPONSE], 0, TFE_OPEN_RESPONSE_CANCEL);
g_signal_emit (tv, tfe_text_view_signals[OPEN_RESPONSE], 0, TFE_OPEN_RESPONSE_ERROR);
\end{lstlisting}

\begin{itemize}
\tightlist
\item
  The first argument is the instance on which the signal is emitted.
\item
  The second argument is the signal id.
\item
  The third argument is the detail of the signal. The signals
  ``change-file'' and ``open-response'' don't have details and the
  arguments are zero.
\item
  The signal ``change-file'' doesn't have parameters, so there's no
  fourth argument.
\item
  The signal ``open-response'' has one parameter. The fourth argument is
  the parameter.
\end{itemize}
