\section{TfeTextView API reference}\label{tfetextview-api-reference}

\subsection{Description}\label{description}

TfeTextView is a child object of GtkTextView. If its contents comes from
a file, it holds the pointer to the GFile. Otherwise, the pointer is
NULL.

\subsection{Hierarchy}\label{hierarchy}

\begin{lstlisting}
GObject -- GInitiallyUnowned -- GtkWidget -- GtkTextView -- TfeTextView
\end{lstlisting}

\subsection{Ancestors}\label{ancestors}

\begin{itemize}
\tightlist
\item
  GtkTextView
\item
  GtkWidget
\item
  GInitiallyUnowned
\item
  GObject
\end{itemize}

\subsection{Constructors}\label{constructors}

\begin{itemize}
\tightlist
\item
  tfe\_text\_view\_new ()
\item
  tfe\_text\_view\_new\_with\_file ()
\end{itemize}

\subsection{Instance methods}\label{instance-methods}

\begin{itemize}
\tightlist
\item
  tfe\_text\_view\_get\_file ()
\item
  tfe\_text\_view\_open ()
\item
  tfe\_text\_view\_save ()
\item
  tfe\_text\_view\_saveas ()
\end{itemize}

\subsection{Signals}\label{signals}

\begin{itemize}
\tightlist
\item
  Tfe.TextView::change-file
\item
  Tfe.TextView::open-response
\end{itemize}

\subsection{API for constructors, instance methods and
signals}\label{api-for-constructors-instance-methods-and-signals}

\textbf{constructors}

\subsubsection{tfe\_text\_view\_new()}\label{tfe_text_view_new}

\begin{lstlisting}
GtkWidget *
tfe_text_view_new (void);
\end{lstlisting}

Creates a new TfeTextView instance and returns the pointer to it as
GtkWidget. If an error happens, it returns
\passthrough{\lstinline!NULL!}.

Return value

\begin{itemize}
\tightlist
\item
  a new TfeTextView.
\end{itemize}

\subsubsection{tfe\_text\_view\_new\_with\_file()}\label{tfe_text_view_new_with_file}

\begin{lstlisting}
GtkWidget *
tfe_text_view_new_with_file (GFile *file);
\end{lstlisting}

Creates a new TfeTextView, reads the contents of the
\passthrough{\lstinline!file!} and set it to the GtkTextBuffer
corresponds to the newly created TfeTextView. Then returns the pointer
to the TfeTextView as GtkWidget. If an error happens, it returns
\passthrough{\lstinline!NULL!}.

Parameters

\begin{itemize}
\tightlist
\item
  file: a GFile
\end{itemize}

Return value

\begin{itemize}
\tightlist
\item
  a new TfeTextView.
\end{itemize}

\textbf{Instance methods}

\subsubsection{tfe\_text\_view\_get\_file()}\label{tfe_text_view_get_file}

\begin{lstlisting}
GFile *
tfe_text_view_get_file (TfeTextView *tv);
\end{lstlisting}

Returns the copy of the GFile in the TfeTextView.

Parameters

\begin{itemize}
\tightlist
\item
  tv: a TfeTextView
\end{itemize}

Return value

\begin{itemize}
\tightlist
\item
  the pointer to the GFile
\end{itemize}

\subsubsection{tfe\_text\_view\_open()}\label{tfe_text_view_open}

\begin{lstlisting}
void
tfe_text_view_open (TfeTextView *tv, GtkWidget *win);
\end{lstlisting}

Shows a file chooser dialog so that a user can choose a file to read.
Then, read the file and set the buffer with the contents. This function
doesn't return the I/O status. Instead, the status is informed by
\passthrough{\lstinline!open-response!} signal. The caller needs to set
a handler to this signal in advance.

parameters

\begin{itemize}
\tightlist
\item
  tv: a TfeTextView
\item
  win: the top level window
\end{itemize}

\subsubsection{tfe\_text\_view\_save()}\label{tfe_text_view_save}

\begin{lstlisting}
void
tfe_text_view_save (TfeTextView *tv);
\end{lstlisting}

Saves the contents of the buffer to the file. If
\passthrough{\lstinline!tv!} holds a GFile, it is used. Otherwise, this
function shows a file chooser dialog so that the user can choose a file
to save.

Parameters

\begin{itemize}
\tightlist
\item
  tv: a TfeTextView
\end{itemize}

\subsubsection{tfe\_text\_view\_saveas()}\label{tfe_text_view_saveas}

\begin{lstlisting}
void
tfe_text_view_saveas (TfeTextView *tv);
\end{lstlisting}

Saves the contents of the buffer to a file. This function shows file
chooser dialog so that a user can choose a file to save.

Parameters

\begin{itemize}
\tightlist
\item
  tv: a TfeTextView
\end{itemize}

\textbf{Signals}

\subsubsection{change-file}\label{change-file}

\begin{lstlisting}
void
user_function (TfeTextView *tv,
               gpointer user_data)
\end{lstlisting}

Emitted when the GFile in the TfeTextView object is changed. The signal
is emitted when:

\begin{itemize}
\tightlist
\item
  a new file is opened and read
\item
  a user chooses a file with the file chooser dialog and save the
  contents.
\end{itemize}

\subsubsection{open-response}\label{open-response}

\begin{lstlisting}
void
user_function (TfeTextView *tv,
               TfeTextViewOpenResponseType response-id,
               gpointer user_data)
\end{lstlisting}

Emitted after the user calls
\passthrough{\lstinline!tfe\_text\_view\_open!}. This signal informs the
status of file I/O operation.

\textbf{Enumerations}

\subsubsection{TfeTextViewOpenResponseType}\label{tfetextviewopenresponsetype}

Predefined values for the response id given by
\passthrough{\lstinline!open-response!} signal.

Members:

\begin{itemize}
\tightlist
\item
  TFE\_OPEN\_RESPONSE\_SUCCESS: The file is successfully opened.
\item
  TFE\_OPEN\_RESPONSE\_CANCEL: Reading file is canceled by the user.
\item
  TFE\_OPEN\_RESPONSE\_ERROR: An error happened during the opening or
  reading process.
\end{itemize}
