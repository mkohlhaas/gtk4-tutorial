\section{Functions in GtkNotebook}\label{functions-in-gtknotebook}

GtkNotebook is a very important object in the text file editor
\passthrough{\lstinline!tfe!}. It connects the application and
TfeTextView objects. A set of public functions are declared in
\passthrough{\lstinline!tfenotebook.h!}. The word ``tfenotebook'' is
used only in filenames. There's no ``TfeNotebook'' object.

The source files are in the directory
\passthrough{\lstinline!src/tfe5!}. You can get them by downloading the
\href{https://github.com/ToshioCP/Gtk4-tutorial}{repository}.

\begin{lstlisting}[language=C, numbers=left]
void
notebook_page_save(GtkNotebook *nb);

void
notebook_page_close (GtkNotebook *nb);

void
notebook_page_open (GtkNotebook *nb);

void
notebook_page_new_with_file (GtkNotebook *nb, GFile *file);

void
notebook_page_new (GtkNotebook *nb);
\end{lstlisting}

This header file describes the public functions in
\passthrough{\lstinline!tfenotebook.c!}.

\begin{itemize}
\tightlist
\item
  1-2: \passthrough{\lstinline!notebook\_page\_save!} saves the current
  page to the file of which the name specified in the tab. If the name
  is \passthrough{\lstinline!untitled!} or
  \passthrough{\lstinline!untitled!} followed by digits, a file chooser
  dialog appears and a user can choose or specify a filename.
\item
  4-5: \passthrough{\lstinline!notebook\_page\_close!} closes the
  current page.
\item
  7-8: \passthrough{\lstinline!notebook\_page\_open!} shows a file
  chooser dialog and a user can choose a file. The contents of the file
  is inserted to a new page.
\item
  10-11: \passthrough{\lstinline!notebook\_page\_new\_with\_file!}
  creates a new page and a file given as an argument is read and
  inserted into the page.
\item
  13-14: \passthrough{\lstinline!notebook\_page\_new!} creates a new
  empty page.
\end{itemize}

You probably find that the functions except
\passthrough{\lstinline!notebook\_page\_close!} are higher level
functions of

\begin{itemize}
\tightlist
\item
  \passthrough{\lstinline!tfe\_text\_view\_save!}
\item
  \passthrough{\lstinline!tef\_text\_view\_open!}
\item
  \passthrough{\lstinline!tfe\_text\_view\_new\_with\_file!}
\item
  \passthrough{\lstinline!tfe\_text\_view\_new!}
\end{itemize}

respectively.

There are two layers. One of them is
\passthrough{\lstinline!tfe\_text\_view ...!}, which is the lower level
layer. The other is \passthrough{\lstinline!notebook ...!}, which is the
higher level layer.

Now let's look at the program of each function.

\subsection{notebook\_page\_new}\label{notebook_page_new}

\begin{lstlisting}[language=C, numbers=left]
static char*
get_untitled () {
  static int c = -1;
  if (++c == 0) 
    return g_strdup_printf("Untitled");
  else
    return g_strdup_printf ("Untitled%u", c);
}

static void
notebook_page_build (GtkNotebook *nb, GtkWidget *tv, const char *filename) {
  GtkWidget *scr = gtk_scrolled_window_new ();
  GtkNotebookPage *nbp;
  GtkWidget *lab;
  int i;

  gtk_scrolled_window_set_child (GTK_SCROLLED_WINDOW (scr), tv);
  lab = gtk_label_new (filename);
  i = gtk_notebook_append_page (nb, scr, lab);
  nbp = gtk_notebook_get_page (nb, scr);
  g_object_set (nbp, "tab-expand", TRUE, NULL);
  gtk_notebook_set_current_page (nb, i);
  g_signal_connect (GTK_TEXT_VIEW (tv), "change-file", G_CALLBACK (file_changed_cb), nb);
}

void
notebook_page_new (GtkNotebook *nb) {
  g_return_if_fail(GTK_IS_NOTEBOOK (nb));

  GtkWidget *tv;
  char *filename;

  tv = tfe_text_view_new ();
  filename = get_untitled ();
  notebook_page_build (nb, tv, filename);
  g_free (filename);
}
\end{lstlisting}

\begin{itemize}
\tightlist
\item
  26-37: The function \passthrough{\lstinline!notebook\_page\_new!}.
\item
  28: The function \passthrough{\lstinline!g\_return\_if\_fail!} checks
  the argument. It's necessary because the function is public.
\item
  33: Creates TfeTextView object.
\item
  34: Creates filename, which is ``Untitled'', ``Untitled1'', \ldots{} .
\item
  1-8: The function \passthrough{\lstinline!get\_untitled!}.
\item
  3: Static variable \passthrough{\lstinline!c!} is initialized at the
  first call of this function. After that \passthrough{\lstinline!c!}
  keeps its value unless it is changed explicitly.
\item
  4-7: Increases \passthrough{\lstinline!c!} by one and if it is zero,
  it returns ``Untitled''. If it is a positive integer, it returns
  ``Untitled\textless the integer\textgreater{}'', for example,
  ``Untitled1'', ``Untitled2'', and so on. The function
  \passthrough{\lstinline!g\_strdup\_printf!} creates a string and it
  should be freed by \passthrough{\lstinline!g\_free!} when it becomes
  useless. The caller of \passthrough{\lstinline!get\_untitled!} is in
  charge of freeing the string.
\item
  36: Calls \passthrough{\lstinline!notebook\_page\_build!} to build a
  new page.
\item
  37: Frees \passthrough{\lstinline!filename!}.
\item
  10- 24: The function \passthrough{\lstinline!notebook\_page\_build!}.
  A parameter with \passthrough{\lstinline!const!} qualifier doesn't
  change in the function. It means that the argument
  \passthrough{\lstinline!filename!} is owned by the caller. The caller
  needs to free it when it becomes useless.
\item
  12: Creates GtkScrolledWindow.
\item
  17: Inserts \passthrough{\lstinline!tv!} to GtkScrolledWindow as a
  child.
\item
  18-19: Creates GtkLabel, then appends \passthrough{\lstinline!scr!}
  and \passthrough{\lstinline!lab!} to the GtkNotebook instance
  \passthrough{\lstinline!nb!}.
\item
  20-21: Sets ``tab-expand'' property to TRUE. The function
  \passthrough{\lstinline!g\_object\_set!} sets properties on an object.
  The object can be any object derived from GObject. In many cases, an
  object has its own function to set its properties, but sometimes
  doesn't. In that case, use \passthrough{\lstinline!g\_object\_set!} to
  set the property.
\item
  22: Sets the current page to the newly created page.
\item
  23: Connects ``change-file'' signal and the handler
  \passthrough{\lstinline!file\_changed\_cb!}.
\end{itemize}

\subsection{notebook\_page\_new\_with\_file}\label{notebook_page_new_with_file}

\begin{lstlisting}[language=C, numbers=left]
void
notebook_page_new_with_file (GtkNotebook *nb, GFile *file) {
  g_return_if_fail(GTK_IS_NOTEBOOK (nb));
  g_return_if_fail(G_IS_FILE (file));

  GtkWidget *tv;
  char *filename;

  if ((tv = tfe_text_view_new_with_file (file)) == NULL)
    return; /* read error */
  filename = g_file_get_basename (file);
  notebook_page_build (nb, tv, filename);
  g_free (filename);
}
\end{lstlisting}

\begin{itemize}
\tightlist
\item
  9-10: Calls
  \passthrough{\lstinline!tfe\_text\_view\_new\_with\_file!}. If the
  function returns NULL, an error has happened. Then, it does nothing
  and returns.
\item
  11-13: Gets the filename, builds a new page and frees
  \passthrough{\lstinline!filename!}.
\end{itemize}

\subsection{notebook\_page\_open}\label{notebook_page_open}

\begin{lstlisting}[language=C, numbers=left]
static void
open_response_cb (TfeTextView *tv, int response, GtkNotebook *nb) {
  GFile *file;
  char *filename;

  if (response != TFE_OPEN_RESPONSE_SUCCESS) {
    g_object_ref_sink (tv);
    g_object_unref (tv);
  }else {
    file = tfe_text_view_get_file (tv);
    filename = g_file_get_basename (file);
    g_object_unref (file);
    notebook_page_build (nb, GTK_WIDGET (tv), filename);
    g_free (filename);
  }
}

void
notebook_page_open (GtkNotebook *nb) {
  g_return_if_fail(GTK_IS_NOTEBOOK (nb));

  GtkWidget *tv;

  tv = tfe_text_view_new ();
  g_signal_connect (TFE_TEXT_VIEW (tv), "open-response", G_CALLBACK (open_response_cb), nb);
  tfe_text_view_open (TFE_TEXT_VIEW (tv), GTK_WINDOW (gtk_widget_get_ancestor (GTK_WIDGET (nb), GTK_TYPE_WINDOW)));
}
\end{lstlisting}

\begin{itemize}
\tightlist
\item
  18-27: The function \passthrough{\lstinline!notebook\_page\_open!}.
\item
  24: Creates TfeTextView object.
\item
  25: Connects the signal ``open-response'' and the handler
  \passthrough{\lstinline!open\_response\_cb!}.
\item
  26: Calls \passthrough{\lstinline!tfe\_text\_view\_open!}. The
  ``open-response'' signal will be emitted later in this function to
  inform the result.
\item
  1-16: The handler \passthrough{\lstinline!open\_response\_cb!}.
\item
  6-8: If the response code is not
  \passthrough{\lstinline!TFE\_OPEN\_RESPONSE\_SUCCESS!}, the instance
  \passthrough{\lstinline!tv!} will be destroyed. It has floating
  reference, which will be explained later. A floating reference needs
  to be converted into an ordinary reference before releasing it. The
  function \passthrough{\lstinline!g\_object\_ref\_sink!} does that.
  After that, the function \passthrough{\lstinline!g\_object\_unref!}
  releases \passthrough{\lstinline!tv!} and decreases the reference
  count by one. Finally the reference count becomes zero and
  \passthrough{\lstinline!tv!} is destroyed.
\item
  9-15: Otherwise, it builds a new page with
  \passthrough{\lstinline!tv!}.
\end{itemize}

\subsection{Floating reference}\label{floating-reference}

All the widgets are derived from GInitiallyUnowned. GObject and
GInitiallyUnowned are almost the same. The difference is like this. When
an instance of GInitiallyUnowned is created, the instance has a
``floating reference''. On the other hand, when an instance of GObject
(not GInitiallyUnowned) is created, it has ``normal reference''. Their
descendants inherits them, so every widget has a floating reference just
after the creation. Non-widget class, for example, GtkTextBuffer is a
direct sub class of GObject and it has normal reference.

The function \passthrough{\lstinline!g\_object\_ref\_sink!} converts the
floating reference into a normal reference. If the instance doesn't have
a floating reference, \passthrough{\lstinline!g\_object\_ref\_sink!}
simply increases the reference count by one. It is used when an widget
is added to another widget as a child.

\begin{lstlisting}
GtkTextView *tv = gtk_text_view_new (); // Floating reference
GtkScrolledWindow *scr = gtk_scrolled_window_new ();
gtk_scrolled_window_set_child (scr, tv); // Scrolled window sink the tv's floating reference and tv's reference count becomes one.
\end{lstlisting}

When \passthrough{\lstinline!tv!} is added to
\passthrough{\lstinline!scr!} as a child,
\passthrough{\lstinline!g\_object\_ref\_sink!} is used.

\begin{lstlisting}
g_object_ref_sink (tv);
\end{lstlisting}

So, the floating reference is converted into an ordinary reference. That
is to say, floating reference is removed, and the normal reference count
is one. Thanks to this, the caller doesn't need to decrease tv's
reference count. If an Object\_A is not a descendant of
GInitiallyUnowned, the program is like this:

\begin{lstlisting}
Object_A *obj_a = object_a_new (); // reference count is one
GtkScrolledWindow *scr = gtk_scrolled_window_new ();
gtk_scrolled_window_set_child (scr, obj_a); // obj_a's reference count is two
// obj_a is referred by the caller (this program) and scrolled window
g_object_unref (obj_a); // obj_a's reference count is one because the caller no longer refers obj_a.
\end{lstlisting}

This example tells us that the caller needs to unref
\passthrough{\lstinline!obj\_a!}.

If you use \passthrough{\lstinline!g\_object\_unref!} to an instance
that has a floating reference, you need to convert the floating
reference to a normal reference in advance. See
\href{https://docs.gtk.org/gobject/floating-refs.html}{GObject API
reference} for further information.

\subsection{notebook\_page\_close}\label{notebook_page_close}

\begin{lstlisting}[language=C, numbers=left]
void
notebook_page_close (GtkNotebook *nb) {
  g_return_if_fail(GTK_IS_NOTEBOOK (nb));

  GtkWidget *win;
  int i;

  if (gtk_notebook_get_n_pages (nb) == 1) {
    win = gtk_widget_get_ancestor (GTK_WIDGET (nb), GTK_TYPE_WINDOW);
    gtk_window_destroy(GTK_WINDOW (win));
  } else {
    i = gtk_notebook_get_current_page (nb);
    gtk_notebook_remove_page (GTK_NOTEBOOK (nb), i);
  }
}
\end{lstlisting}

This function closes the current page. If the page is the only page the
notebook has, then the function destroys the top-level window and quits
the application.

\begin{itemize}
\tightlist
\item
  8-10: If the page is the only page the notebook has, it calls
  \passthrough{\lstinline!gtk\_window\_destroy!} to destroy the
  top-level window.
\item
  11-13: Otherwise, removes the current page. The child widget
  (TfeTextView) is also destroyed.
\end{itemize}

\subsection{notebook\_page\_save}\label{notebook_page_save}

\begin{lstlisting}[language=C, numbers=left]
static TfeTextView *
get_current_textview (GtkNotebook *nb) {
  int i;
  GtkWidget *scr;
  GtkWidget *tv;

  i = gtk_notebook_get_current_page (nb);
  scr = gtk_notebook_get_nth_page (nb, i);
  tv = gtk_scrolled_window_get_child (GTK_SCROLLED_WINDOW (scr));
  return TFE_TEXT_VIEW (tv);
}

void
notebook_page_save (GtkNotebook *nb) {
  g_return_if_fail(GTK_IS_NOTEBOOK (nb));

  TfeTextView *tv;

  tv = get_current_textview (nb);
  tfe_text_view_save (tv);
}
\end{lstlisting}

\begin{itemize}
\tightlist
\item
  13-21: \passthrough{\lstinline!notebook\_page\_save!}.
\item
  19: Gets the TfeTextView instance belongs to the current page. The
  caller doesn't have the ownership of \passthrough{\lstinline!tv!} so
  you don't need to care about freeing it.
\item
  20: Calls \passthrough{\lstinline!tfe\_text\_view\_save!}.
\item
  1-11: \passthrough{\lstinline!get\_current\_textview!}. This function
  gets the TfeTextView object belongs to the current page.
\item
  7: Gets the page number of the current page.
\item
  8: Gets the child widget \passthrough{\lstinline!scr!}, which is a
  GtkScrolledWindow instance, of the current page. The object
  \passthrough{\lstinline!scr!} is owned by the notebook
  \passthrough{\lstinline!nb!}. So, the caller doesn't need to free it.
\item
  9-10: Gets the child widget of \passthrough{\lstinline!scr!}, which is
  a TfeTextView instance, and returns it. The returned instance is owned
  by \passthrough{\lstinline!scr!} and the caller of
  \passthrough{\lstinline!get\_cuurent\_textview!} doesn't need to care
  about freeing it.
\end{itemize}

\subsection{file\_changed\_cb handler}\label{file_changed_cb-handler}

The function \passthrough{\lstinline!file\_changed\_cb!} is a handler
connected to ``change-file'' signal. If a file in a TfeTextView instance
is changed, the instance emits this signal. This handler changes the
label of the GtkNotebookPage.

\begin{lstlisting}[language=C, numbers=left]
static void
file_changed_cb (TfeTextView *tv, GtkNotebook *nb) {
  GtkWidget *scr;
  GtkWidget *label;
  GFile *file;
  char *filename;

  file = tfe_text_view_get_file (tv);
  scr = gtk_widget_get_parent (GTK_WIDGET (tv));
  if (G_IS_FILE (file)) {
    filename = g_file_get_basename (file);
    g_object_unref (file);
  } else
    filename = get_untitled ();
  label = gtk_label_new (filename);
  g_free (filename);
  gtk_notebook_set_tab_label (nb, scr, label);
}
\end{lstlisting}

\begin{itemize}
\tightlist
\item
  8: Gets the GFile instance from \passthrough{\lstinline!tv!}.
\item
  9: Gets the GkScrolledWindow instance which is the parent widget of
  \passthrough{\lstinline!tv!}.
\item
  10-12: If \passthrough{\lstinline!file!} points a GFile instance, the
  filename of the GFile is assigned to
  \passthrough{\lstinline!filename!}. Then, unref the GFile object
  \passthrough{\lstinline!file!}.
\item
  13-14: Otherwise (file is NULL), a string
  \passthrough{\lstinline!Untitled(number)!} is assigned to
  \passthrough{\lstinline!filename!}.
\item
  15-17: Creates a GtkLabel instance \passthrough{\lstinline!label!}
  with the filename and set the label of the GtkNotebookPage with
  \passthrough{\lstinline!label!}.
\end{itemize}
