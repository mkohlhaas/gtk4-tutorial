\section{Pango, CSS and Application}\label{pango-css-and-application}

\subsection{PangoFontDescription}\label{pangofontdescription}

PangoFontDescription is a C structure for a font. You can get font
family, style, weight and size. You can also get a string that includes
font attributes. For example, suppose that the PangoFontDescription has
a font of ``Noto Sans Mono'', ``Bold'', ``Italic'' and 12 points of
size. Then the string converted from the PangoFontDescription is ``Noto
Sans Mono Bold Italic 12''.

\begin{itemize}
\tightlist
\item
  Font family is ``Noto Sans Mono''.
\item
  Font style is ``Italic''.
\item
  Font weight is ``Bold'', or 700.
\item
  Font size is 12 pt.
\end{itemize}

The font in CSS is different from the string from PangoFontDescription.

\begin{itemize}
\tightlist
\item
  \passthrough{\lstinline!font: bold italic 12pt "Noto Sans Mono"!}
\item
  \passthrough{\lstinline!Noto Sans Mono Bold Italic 12!}
\end{itemize}

So, it may be easier to use each property, i.e.~font-family, font-style,
font-weight and font-size, to convert a PangoFontDescription data to
CSS.

Refer to \href{https://docs.gtk.org/Pango/index.html}{Pango document}
and \href{https://www.w3.org/TR/css-fonts-3/}{W3C CSS Fonts Module Level
3} for further information.

\subsection{Converter from PangoFontDescription to
CSS}\label{converter-from-pangofontdescription-to-css}

Two files \passthrough{\lstinline!pfd2css.h!} and
\passthrough{\lstinline!pfd2css.c!} include the converter from
PangoFontDescription to CSS.

\begin{lstlisting}[language=C, numbers=left]
#pragma once

#include <pango/pango.h>

// Pango font description to CSS style string
// Returned string is owned by the caller. The caller should free it when it becomes useless.

char*
pfd2css (PangoFontDescription *pango_font_desc);

// Each element (family, style, weight and size)

const char*
pfd2css_family (PangoFontDescription *pango_font_desc);

const char*
pfd2css_style (PangoFontDescription *pango_font_desc);

int
pfd2css_weight (PangoFontDescription *pango_font_desc);

// Returned string is owned by the caller. The caller should free it when it becomes useless.
char *
pfd2css_size (PangoFontDescription *pango_font_desc);
\end{lstlisting}

The five functions are public. The first function is a convenient
function to set other four CSS at once.

\begin{lstlisting}[language=C, numbers=left]
#include <pango/pango.h>
#include "pfd2css.h"

// Pango font description to CSS style string
// Returned string is owned by caller. The caller should free it when it is useless.

char*
pfd2css (PangoFontDescription *pango_font_desc) {
  char *fontsize;

  fontsize = pfd2css_size (pango_font_desc);
  return g_strdup_printf ("font-family: \"%s\"; font-style: %s; font-weight: %d; font-size: %s;",
              pfd2css_family (pango_font_desc), pfd2css_style (pango_font_desc),
              pfd2css_weight (pango_font_desc), fontsize);
  g_free (fontsize); 
}

// Each element (family, style, weight and size)

const char*
pfd2css_family (PangoFontDescription *pango_font_desc) {
  return pango_font_description_get_family (pango_font_desc);
}

const char*
pfd2css_style (PangoFontDescription *pango_font_desc) {
  PangoStyle pango_style = pango_font_description_get_style (pango_font_desc);
  switch (pango_style) {
  case PANGO_STYLE_NORMAL:
    return "normal";
  case PANGO_STYLE_ITALIC:
    return "italic";
  case PANGO_STYLE_OBLIQUE:
    return "oblique";
  default:
    return "normal";
  }
}

int
pfd2css_weight (PangoFontDescription *pango_font_desc) {
  PangoWeight pango_weight = pango_font_description_get_weight (pango_font_desc);
  switch (pango_weight) {
  case PANGO_WEIGHT_THIN:
    return 100;
  case PANGO_WEIGHT_ULTRALIGHT:
    return 200;
  case PANGO_WEIGHT_LIGHT:
    return 300;
  case PANGO_WEIGHT_SEMILIGHT:
    return 350;
  case PANGO_WEIGHT_BOOK:
    return 380;
  case PANGO_WEIGHT_NORMAL:
    return 400; /* or "normal" */
  case PANGO_WEIGHT_MEDIUM:
    return 500;
  case PANGO_WEIGHT_SEMIBOLD:
    return 600;
  case PANGO_WEIGHT_BOLD:
    return 700; /* or "bold" */
  case PANGO_WEIGHT_ULTRABOLD:
    return 800;
  case PANGO_WEIGHT_HEAVY:
    return 900;
  case PANGO_WEIGHT_ULTRAHEAVY:
    return 900; /* 1000 is available since CSS Fonts level 4 but GTK currently supports level 3. */
  default:
    return 400; /* "normal" */
  }
}

char *
pfd2css_size (PangoFontDescription *pango_font_desc) {
  if (pango_font_description_get_size_is_absolute (pango_font_desc))
    return g_strdup_printf ("%dpx", pango_font_description_get_size (pango_font_desc) / PANGO_SCALE);
  else
    return g_strdup_printf ("%dpt", pango_font_description_get_size (pango_font_desc) / PANGO_SCALE);
}
\end{lstlisting}

\begin{itemize}
\tightlist
\item
  The function \passthrough{\lstinline!pfd2css\_family!} returns font
  family.
\item
  The function \passthrough{\lstinline!pfd2css\_style!} returns font
  style which is one of ``normal'', ``italic'' or ``oblique''.
\item
  The function \passthrough{\lstinline!pfd2css\_weight!} returns font
  weight in integer. See the list below.
\item
  The function \passthrough{\lstinline!pfd2css\_size!} returns font
  size.

  \begin{itemize}
  \tightlist
  \item
    If the font description size is absolute, it returns the size of
    device unit, which is pixel. Otherwise the unit is point.
  \item
    The function
    \passthrough{\lstinline!pango\_font\_description\_get\_size!}
    returns the integer of the size but it is multiplied by
    \passthrough{\lstinline!PANGO\_SCALE!}. So, you need to divide it by
    \passthrough{\lstinline!PANGO\_SCALE!}. The
    \passthrough{\lstinline!PANGO\_SCALE!} is currently 1024, but this
    might be changed in the future. If the font size is 12pt, the size
    in pango is \passthrough{\lstinline!12*PANGO\_SCALE=12*1024=12288!}.
  \end{itemize}
\item
  The function \passthrough{\lstinline!pfd2css!} returns a string of the
  font. For example, if a font ``Noto Sans Mono Bold Italic 12'' is
  given, it returns ``font-family: Noto Sans Mono; font-style: italic;
  font-weight: 700; font-size: 12pt;''.
\end{itemize}

The font weight number is one of:

\begin{itemize}
\tightlist
\item
  100 - Thin
\item
  200 - Extra Light (Ultra Light)
\item
  300 - Light
\item
  400 - Normal
\item
  500 - Medium
\item
  600 - Semi Bold (Demi Bold)
\item
  700 - Bold
\item
  800 - Extra Bold (Ultra Bold)
\item
  900 - Black (Heavy)
\end{itemize}

\subsection{Application object}\label{application-object}

\subsubsection{TfeApplication class}\label{tfeapplication-class}

TfeApplication class is a child of GtkApplication. It has some instance
variables. The header file defines the type macro and a public function.

\begin{lstlisting}[language=C, numbers=left]
#pragma once

#include <gtk/gtk.h>

#define TFE_TYPE_APPLICATION tfe_application_get_type ()
G_DECLARE_FINAL_TYPE (TfeApplication, tfe_application, TFE, APPLICATION, GtkApplication)

TfeApplication *
tfe_application_new (const char* application_id, GApplicationFlags flag);
\end{lstlisting}

The following code is extracted from
\passthrough{\lstinline!tfeapplication.c!}. It builds TfeApplication
class and instance.

\begin{lstlisting}[language=C]
#include <gtk/gtk.h>
#include "tfeapplication.h"

struct _TfeApplication {
  GtkApplication parent;
  TfeWindow *win;
  GSettings *settings;
  GtkCssProvider *provider;
};

G_DEFINE_FINAL_TYPE (TfeApplication, tfe_application, GTK_TYPE_APPLICATION)

static void
tfe_application_dispose (GObject *gobject) {
  TfeApplication *app = TFE_APPLICATION (gobject);

  g_clear_object (&app->settings);
  g_clear_object (&app->provider);
  G_OBJECT_CLASS (tfe_application_parent_class)->dispose (gobject);
}

static void
tfe_application_init (TfeApplication *app) {
  app->settings = g_settings_new ("com.github.ToshioCP.tfe");
  g_signal_connect (app->settings, "changed::font-desc", G_CALLBACK (changed_font_cb), app);
  app->provider = gtk_css_provider_new ();
}

static void
tfe_application_class_init (TfeApplicationClass *class) {
  G_OBJECT_CLASS (class)->dispose = tfe_application_dispose;
  G_APPLICATION_CLASS (class)->startup = app_startup;
  G_APPLICATION_CLASS (class)->activate = app_activate;
  G_APPLICATION_CLASS (class)->open = app_open;
}

TfeApplication *
tfe_application_new (const char* application_id, GApplicationFlags flag) {
  return TFE_APPLICATION (g_object_new (TFE_TYPE_APPLICATION, "application-id", application_id, "flags", flag, NULL));
}
\end{lstlisting}

\begin{itemize}
\tightlist
\item
  The structure \passthrough{\lstinline!\_TfeApplication!} is defined.
  It has four members. One is its parents and the others are kinds of
  instance variables. The members are usually initialized in the
  instance initialization function. And they are released in the
  disposal function or freed in the finalization function. The members
  are:

  \begin{itemize}
  \tightlist
  \item
    win: main window instance
  \item
    settings: GSettings instance.it is bound to ``font-desc'' item in
    the GSettings
  \item
    provider: a provider for the font of the textview.
  \end{itemize}
\item
  The macro \passthrough{\lstinline!G\_DEFINE\_FINAL\_TYPE!} defines
  \passthrough{\lstinline!tfe\_application\_get\_type!} function and
  some other useful things.
\item
  The function \passthrough{\lstinline!tfe\_application\_class\_init!}
  initializes the TfeApplication class. It overrides four class methods.
  Three class methods \passthrough{\lstinline!startup!},
  \passthrough{\lstinline!activate!} and \passthrough{\lstinline!open!}
  points the default signal handlers. The overriding changes the default
  handlers. You can connect the handlers with
  \passthrough{\lstinline!g\_signal\_connect!} macro but the result is
  different. The macro connects a user handler to the signal. The
  default handler still exists and no change is made to them.
\item
  The function \passthrough{\lstinline!tfe\_application\_init!}
  initializes an instance.

  \begin{itemize}
  \tightlist
  \item
    Creates a new GSettings instance and make
    \passthrough{\lstinline!app->settings!} point it. Then connects the
    handler \passthrough{\lstinline!changed\_font\_cb!} to the
    ``changed::font-desc'' signal.
  \item
    Creates a new GtkCssProvider instance and make
    \passthrough{\lstinline!app->provider!} point it.
  \end{itemize}
\item
  The function \passthrough{\lstinline!tfe\_application\_dispose!}
  releases the GSettings and GtkCssProvider instances. Then, call the
  parent's dispose handler. It is called ``chaining up''. See
  \href{https://docs.gtk.org/gobject/tutorial.html\#chaining-up}{GObject
  document}.
\end{itemize}

\subsubsection{Startup signal handlers}\label{startup-signal-handlers}

\begin{lstlisting}[language=C, numbers=left]
static void
app_startup (GApplication *application) {
  TfeApplication *app = TFE_APPLICATION (application);
  int i;
  GtkCssProvider *provider = gtk_css_provider_new ();
  GdkDisplay *display;

  G_APPLICATION_CLASS (tfe_application_parent_class)->startup (application);

  app->win = TFE_WINDOW (tfe_window_new (GTK_APPLICATION (app)));

  gtk_css_provider_load_from_data (provider, "textview {padding: 10px;}", -1);
  display = gdk_display_get_default ();
  gtk_style_context_add_provider_for_display (display, GTK_STYLE_PROVIDER (provider),
                                              GTK_STYLE_PROVIDER_PRIORITY_APPLICATION);
  g_object_unref (provider);
  gtk_style_context_add_provider_for_display (display, GTK_STYLE_PROVIDER (app->provider),
                                              GTK_STYLE_PROVIDER_PRIORITY_APPLICATION);

  changed_font_cb (app->settings, "font-desc", app); // Sets the text view font to the font from the gsettings data base.

/* ----- accelerator ----- */ 
  struct {
    const char *action;
    const char *accels[2];
  } action_accels[] = {
    { "win.open", { "<Control>o", NULL } },
    { "win.save", { "<Control>s", NULL } },
    { "win.close", { "<Control>w", NULL } },
    { "win.new", { "<Control>n", NULL } },
    { "win.saveas", { "<Shift><Control>s", NULL } },
    { "win.close-all", { "<Control>q", NULL } },
  };

  for (i = 0; i < G_N_ELEMENTS(action_accels); i++)
    gtk_application_set_accels_for_action(GTK_APPLICATION(app), action_accels[i].action, action_accels[i].accels);
}
\end{lstlisting}

The function \passthrough{\lstinline!app\_startup!} replace the default
signal handlers. It does five things.

\begin{itemize}
\tightlist
\item
  Calls the parent's startup handler. It is called ``chaining up''. The
  ``startup'' default handler runs before user handlers. So the call for
  the parent's handler must be done at the beginning.
\item
  Creates the main window. This application has only one top level
  window. In that case, it is a good way to create the window in the
  startup handler, which is called only once. Activate or open handlers
  can be called twice or more. Therefore, if you create a window in the
  activate or open handler, two or more windows can be created.
\item
  Sets the default display CSS to ``textview \{padding: 10px;\}''. It
  sets the GtkTextView, or TfeTextView, padding to 10px and makes the
  text easier to read. This CSS is fixed and never changed through the
  application life.
\item
  Adds another CSS provider, which is pointed by
  \passthrough{\lstinline!app->provider!}, to the default display. This
  CSS depends on the GSettings ``font-desc'' value and it can be changed
  during the application life time. And calls
  \passthrough{\lstinline!changed\_font\_cb!} to update the font CSS
  setting.
\item
  Sets application accelerator with the function
  \passthrough{\lstinline!gtk\_application\_set\_accels\_for\_action!}.
  Accelerators are kinds of short cut key functions. For example,
  \passthrough{\lstinline!Ctrl+O!} is an accelerator to activate
  ``open'' action. Accelerators are written in the array
  \passthrough{\lstinline!action-accels[]!}. Its element is a structure
  \passthrough{\lstinline!struct \{const char *action; const char *accels[2];\}!}.
  The member \passthrough{\lstinline!action!} is an action name. The
  member \passthrough{\lstinline!accels!} is an array of two pointers.
  For example,
  \passthrough{\lstinline!\{"win.open", \{ "<Control>o", NULL \}\}!}
  tells that the accelerator \passthrough{\lstinline!Ctrl+O!} is
  connected to the ``win.open'' action. The second element of
  \passthrough{\lstinline!accels!} is NULL which is the end mark. You
  can define more than one accelerator keys and the list must ends with
  NULL (zero). If you want to do so, the array length needs to be three
  or more. For example,
  \passthrough{\lstinline!\{"win.open", \{ "<Control>o", "<Alt>o", NULL \}\}!}
  means two accelerators \passthrough{\lstinline!Ctrl+O!} and
  \passthrough{\lstinline!Alt+O!} is connected to the ``win.open''
  action. The parser recognizes ``\textless control\textgreater o'',
  ``\textless Shift\textgreater\textless Alt\textgreater F2'',
  ``\textless Ctrl\textgreater minus'' and so on. If you want to use
  symbol key like ``\textless Ctrl\textgreater-'', use
  ``\textless Ctrl\textgreater minus'' instead. Such relation between
  lower case and symbol (character code) is specified in
  \href{https://gitlab.gnome.org/GNOME/gtk/-/blob/master/gdk/gdkkeysyms.h}{\passthrough{\lstinline!gdkkeysyms.h!}}
  in the GTK 4 source code.
\end{itemize}

\subsubsection{Activate and open signal
handlers}\label{activate-and-open-signal-handlers}

Two functions \passthrough{\lstinline!app\_activate!} and
\passthrough{\lstinline!app\_open!} replace the default signal handlers.

\begin{lstlisting}[language=C, numbers=left]
static void
app_activate (GApplication *application) {
  TfeApplication *app = TFE_APPLICATION (application);

  tfe_window_notebook_page_new (app->win);
  gtk_window_present (GTK_WINDOW (app->win));
}

static void
app_open (GApplication *application, GFile ** files, gint n_files, const gchar *hint) {
  TfeApplication *app = TFE_APPLICATION (application);

  tfe_window_notebook_page_new_with_files (app->win, files, n_files);
  gtk_window_present (GTK_WINDOW (app->win));
}
\end{lstlisting}

The original default handlers don't do useful works and you don't need
to chain up to the parent's default handlers. They just create notebook
pages and show the top level window.

\subsubsection{CSS font setting}\label{css-font-setting}

\begin{lstlisting}[language=C, numbers=left]
static void
changed_font_cb (GSettings *settings, char *key, gpointer user_data) {
  TfeApplication *app = TFE_APPLICATION (user_data);
  char *font, *s, *css;
  PangoFontDescription *pango_font_desc;

  if (g_strcmp0(key, "font-desc") != 0)
    return;
  font = g_settings_get_string (app->settings, "font-desc");
  pango_font_desc = pango_font_description_from_string (font);
  g_free (font);
  s = pfd2css (pango_font_desc); // converts Pango Font Description into CSS style string
  pango_font_description_free (pango_font_desc);
  css = g_strdup_printf ("textview {%s}", s);
  gtk_css_provider_load_from_data (app->provider, css, -1);
  g_free (s);
  g_free (css);
}
\end{lstlisting}

The function \passthrough{\lstinline!changed\_font\_cb!} is a handler
for ``changed::font-desc'' signal on the GSettings instance. The signal
name is ``changed'' and ``font-desc'' is a key name. This signal is
emitted when the value of the ``font-desc'' key is changed. The value is
bound to the ``font-desc'' property of the GtkFontDialogButton instance.
Therefore, the handler \passthrough{\lstinline!changed\_font\_cb!} is
called when the user selects and updates a font through the font dialog.

A string is retrieved from the GSetting database and converts it into a
pango font description. And a CSS string is made by the function
\passthrough{\lstinline!pfd2css!} and
\passthrough{\lstinline!g\_strdup\_printf!}. Then the css provider is
set to the string. The provider has been inserted to the current display
in advance. So, the font is applied to the display.

\subsection{Other files}\label{other-files}

main.c

\begin{lstlisting}[language=C, numbers=left]
#include <gtk/gtk.h>
#include "tfeapplication.h"

#define APPLICATION_ID "com.github.ToshioCP.tfe"

int
main (int argc, char **argv) {
  TfeApplication *app;
  int stat;

  app = tfe_application_new (APPLICATION_ID, G_APPLICATION_HANDLES_OPEN);
  stat =g_application_run (G_APPLICATION (app), argc, argv);
  g_object_unref (app);
  return stat;
}
\end{lstlisting}

Resource XML file.

\begin{lstlisting}[language=XML, numbers=left]
<?xml version="1.0" encoding="UTF-8"?>
<gresources>
  <gresource prefix="/com/github/ToshioCP/tfe">
    <file>tfewindow.ui</file>
    <file>tfepref.ui</file>
    <file>tfealert.ui</file>
    <file>menu.ui</file>
  </gresource>
</gresources>
\end{lstlisting}

GSchema XML file

\begin{lstlisting}[language=XML, numbers=left]
<?xml version="1.0" encoding="UTF-8"?>
<schemalist>
  <schema path="/com/github/ToshioCP/tfe/" id="com.github.ToshioCP.tfe">
    <key name="font-desc" type="s">
      <default>'Monospace 12'</default>
      <summary>Font</summary>
      <description>A font to be used for textview.</description>
    </key>
  </schema>
</schemalist>
\end{lstlisting}

Meson.build

\begin{lstlisting}[numbers=left]
project('tfe', 'c', license : 'GPL-3.0-or-later', meson_version:'>=1.0.1', version: '0.5')

gtkdep = dependency('gtk4')

gnome = import('gnome')
resources = gnome.compile_resources('resources','tfe.gresource.xml')
gnome.compile_schemas(depend_files: 'com.github.ToshioCP.tfe.gschema.xml')

sourcefiles = files('main.c', 'tfeapplication.c', 'tfewindow.c', 'tfepref.c', 'tfealert.c', 'pfd2css.c', '../tfetextview/tfetextview.c')

executable(meson.project_name(), sourcefiles, resources, dependencies: gtkdep, export_dynamic: true, install: true)

schema_dir = get_option('prefix') / get_option('datadir') / 'glib-2.0/schemas/'
install_data('com.github.ToshioCP.tfe.gschema.xml', install_dir: schema_dir)
gnome.post_install (glib_compile_schemas: true)
\end{lstlisting}

\begin{itemize}
\tightlist
\item
  The function \passthrough{\lstinline!project!} defines project and
  initialize meson. The first argument is the project name and the
  second is the language name. The other arguments are keyword
  arguments.
\item
  The function \passthrough{\lstinline!dependency!} defines the
  denpendent library. Tfe depends GTK4. This is used to create
  \passthrough{\lstinline!pkg-config!} option in the command line of C
  compiler to include header files and link libraries. The returned
  object \passthrough{\lstinline!gtkdep!} is used as an argument to the
  \passthrough{\lstinline!executable!} function later.
\item
  The function \passthrough{\lstinline!import!} imports an extension
  module. The GNOME module has some convenient methods like
  \passthrough{\lstinline!gnome.compile\_resources!} and
  \passthrough{\lstinline!gnome.compile\_schemas!}.
\item
  The method \passthrough{\lstinline!gnome.compile\_resources!} compiles
  and creates resource files. The first argument is the resource name
  without extension and the second is the name of XML file. The returned
  value is an array
  \passthrough{\lstinline!['resources,c', 'resources.h']!}.
\item
  The function \passthrough{\lstinline!gnome.compile\_schemas!} compiles
  the schema files in the current directory. This just creates
  \passthrough{\lstinline!gschemas.compiled!} in the build directory. It
  is used to test the executable binary in the build directory. The
  function doesn't install the schema file.
\item
  The function \passthrough{\lstinline!files!} creates a File Object.
\item
  The function \passthrough{\lstinline!executable!} defines the
  compilation elements such as target name, source files, dependencies
  and installation. The target name is ``tfe''. The source files are
  elements of `sourcefiles' and `resources'. It uses GTK4 libraries. It
  can be installed.
\item
  The last three lines are post install work. The variable
  \passthrough{\lstinline!schema\_dir!} is the directory stored the
  schema file. If meson runs with
  \passthrough{\lstinline!--prefix=$HOME/.local!} argument, it is
  \passthrough{\lstinline!$HOME/.local/share/glib-2.9/schemas!}. The
  function \passthrough{\lstinline!install\_data!} copies the first
  argument file into the second argument directory. The method
  \passthrough{\lstinline!gnome.post\_install!} runs
  \passthrough{\lstinline!glib-compile-schemas!} and updates
  \passthrough{\lstinline!gschemas\_compiled!} file.
\end{itemize}

\subsection{Compilation and
installation.}\label{compilation-and-installation.}

If you want to install it to your local area, use
\passthrough{\lstinline!--prefix=$HOME/.local!} or
\passthrough{\lstinline!--prefix=$HOME!} option. If you want to install
it to the system area, no option is needed. It will be installed under
\passthrough{\lstinline!/user/local!} directory.

\begin{lstlisting}
$ meson setup --prefix=$HOME/.local _build
$ ninja -C _build
$ ninja -C _build install
\end{lstlisting}

You need root privilege to install it to the system area..

\begin{lstlisting}
$ meson setup _build
$ ninja -C _build
$ sudo ninja -C _build install
\end{lstlisting}

Source files are in src/tfe6 directory.

We made a very small text editor. You can add features to this editor.
When you add a new feature, be careful about the structure of the
program. Maybe you need to divide a file into several files. It isn't
good to put many things into one file. And it is important to think
about the relationship between source files and widget structures.

The source files are in the
\href{https://github.com/ToshioCP/Gtk4-tutorial}{Gtk4 tutorial GitHub
repository}. Download it and see \passthrough{\lstinline!src/tfe6!}
directory.

Note: When the menu button is clicked, error messages are printed.

\begin{lstlisting}
(tfe:31153): Gtk-CRITICAL **: 13:05:40.746: _gtk_css_corner_value_get_x: assertion 'corner->class == &GTK_CSS_VALUE_CORNER' failed
\end{lstlisting}

I found a
\href{https://discourse.gnome.org/t/menu-button-gives-error-messages-with-latest-gtk4/15689}{message}
in the GNOME Discourse. The comment says that GTK 4.10 has a bug and it
is fixed in the version 4.10.5. I haven't check 4.10.5 yet, where the
UBUNTU GTK4 is still 4.10.4.
