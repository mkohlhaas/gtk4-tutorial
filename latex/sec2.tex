\section{Preparation}\label{preparation}

\subsection{Installing GTK 4 into Linux
distributions}\label{installing-gtk-4-into-linux-distributions}

This section describes how to install GTK 4 into Linux distributions.

There are two ways to install GTK 4.

\begin{itemize}
\tightlist
\item
  Install it from the distribution packages.
\item
  Build it from the source files.
\end{itemize}

\subsubsection{Installation from the distribution
packages}\label{installation-from-the-distribution-packages}

This is the easiest way to install. I've installed GTK 4 packages
(version 4.14.2) on Ubuntu 24.04 LTS.

\begin{lstlisting}
$ sudo apt install libgtk-4-dev
\end{lstlisting}

It is important to install the development files package (libgtk-4-dev).
Otherwise, you can't compile any GTK 4 based programs.

Fedora, Debian, Arch, Gentoo and OpenSUSE also have GTK 4 packages. See
the website of your distribution for further information.

Package information for Arch, Debian/Ubuntu and Fedora is described in
GTK website,
\href{https://www.gtk.org/docs/installations/linux\#installing-gtk-from-packages}{Installing
GTK from packages}.

\subsubsection{Installation from the source
file}\label{installation-from-the-source-file}

If you want to install a developing version of GTK 4, you need to build
it from the source. See
\href{https://docs.gtk.org/gtk4/building.html}{Compiling the GTK
Libraries} section in the GTK 4 API reference.

\subsection{How to download this
repository}\label{how-to-download-this-repository}

There are two ways: zip and git. Downloading a zip file is the easiest
way. However, if you use git and clone this repository, you can easily
update your local repository by \passthrough{\lstinline!git pull!}
command.

\subsubsection{Download a zip file}\label{download-a-zip-file}

\begin{itemize}
\tightlist
\item
  Run your browser and open
  \href{https://github.com/ToshioCP/Gtk4-tutorial}{this repository}.
\item
  Click on the green button with \passthrough{\lstinline!<> Code!}. Then
  a popup menu appears. Click on \passthrough{\lstinline!Download ZIP!}
  menu.
\item
  Then the repository data is downloaded as a zip file into your
  download folder.
\item
  Unzip the file.
\end{itemize}

\subsubsection{Clone the repository}\label{clone-the-repository}

\begin{itemize}
\tightlist
\item
  Click on the green button with the label
  \passthrough{\lstinline!<> Code!}. Then a popup menu appears. The
  first section is \passthrough{\lstinline!Clone!} with three tabs.
  Click \passthrough{\lstinline!HTTPS!} tab and click on the copy icon,
  which is on the right of
  \passthrough{\lstinline!https://github.com/ToshioCP/Gtk4-tutorial.git!}.
\item
  Run your terminal and type \passthrough{\lstinline!git clone!},
  Ctrl+Shift+V. Then the line will be
  \passthrough{\lstinline!git clone https://github.com/ToshioCP/Gtk4-tutorial.git!}.
  Press the enter key.
\item
  A directory \passthrough{\lstinline!Gtk4-tutorial!} is created. It is
  the copy of this repository.
\end{itemize}

\subsection{Examples in the tutorial}\label{examples-in-the-tutorial}

Examples are under the \passthrough{\lstinline!src!} directory. For
example, the first example of the tutorial is
\passthrough{\lstinline!pr1.c!} and its pathname is
\passthrough{\lstinline!src/misc/pr1.c!}. So you don't need to type the
example codes by yourself.
