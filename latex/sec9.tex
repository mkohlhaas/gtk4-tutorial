\section{GtkBuilder and UI file}\label{gtkbuilder-and-ui-file}

\subsection{New, Open and Save button}\label{new-open-and-save-button}

We made very simple editor in the previous section. It reads files at
the start and writes them out at the end of the program. It works, but
is not so good. It would be better if we had ``New'', ``Open'', ``Save''
and ``Close'' buttons. This section describes how to put those buttons
into the window.

\begin{figure}
\centering
\includegraphics[width=9.3cm,height=6.825cm]{../image/screenshot_tfe2.png}
\caption{Screenshot of the file editor}
\end{figure}

The screenshot above shows the layout. The function
\passthrough{\lstinline!app\_open!} in the source code
\passthrough{\lstinline!tfe2.c!} is as follows.

\begin{lstlisting}[language=C, numbers=left]
static void
app_open (GApplication *app, GFile ** files, gint n_files, gchar *hint) {
  GtkWidget *win;
  GtkWidget *nb;
  GtkWidget *lab;
  GtkNotebookPage *nbp;
  GtkWidget *scr;
  GtkWidget *tv;
  GtkTextBuffer *tb;
  char *contents;
  gsize length;
  char *filename;
  int i;
  GError *err = NULL;

  GtkWidget *boxv;
  GtkWidget *boxh;
  GtkWidget *dmy1;
  GtkWidget *dmy2;
  GtkWidget *dmy3;
  GtkWidget *btnn; /* button for new */
  GtkWidget *btno; /* button for open */
  GtkWidget *btns; /* button for save */
  GtkWidget *btnc; /* button for close */

  win = gtk_application_window_new (GTK_APPLICATION (app));
  gtk_window_set_title (GTK_WINDOW (win), "file editor");
  gtk_window_set_default_size (GTK_WINDOW (win), 600, 400);

  boxv = gtk_box_new (GTK_ORIENTATION_VERTICAL, 0);
  gtk_window_set_child (GTK_WINDOW (win), boxv);

  boxh = gtk_box_new (GTK_ORIENTATION_HORIZONTAL, 0);
  gtk_box_append (GTK_BOX (boxv), boxh);

  dmy1 = gtk_label_new(NULL); /* dummy label for left space */
  gtk_label_set_width_chars (GTK_LABEL (dmy1), 10);
  dmy2 = gtk_label_new(NULL); /* dummy label for center space */
  gtk_widget_set_hexpand (dmy2, TRUE);
  dmy3 = gtk_label_new(NULL); /* dummy label for right space */
  gtk_label_set_width_chars (GTK_LABEL (dmy3), 10);
  btnn = gtk_button_new_with_label ("New");
  btno = gtk_button_new_with_label ("Open");
  btns = gtk_button_new_with_label ("Save");
  btnc = gtk_button_new_with_label ("Close");

  gtk_box_append (GTK_BOX (boxh), dmy1);
  gtk_box_append (GTK_BOX (boxh), btnn);
  gtk_box_append (GTK_BOX (boxh), btno);
  gtk_box_append (GTK_BOX (boxh), dmy2);
  gtk_box_append (GTK_BOX (boxh), btns);
  gtk_box_append (GTK_BOX (boxh), btnc);
  gtk_box_append (GTK_BOX (boxh), dmy3);

  nb = gtk_notebook_new ();
  gtk_widget_set_hexpand (nb, TRUE);
  gtk_widget_set_vexpand (nb, TRUE);
  gtk_box_append (GTK_BOX (boxv), nb);

  for (i = 0; i < n_files; i++) {
    if (g_file_load_contents (files[i], NULL, &contents, &length, NULL, &err)) {
      scr = gtk_scrolled_window_new ();
      tv = tfe_text_view_new ();
      tb = gtk_text_view_get_buffer (GTK_TEXT_VIEW (tv));
      gtk_text_view_set_wrap_mode (GTK_TEXT_VIEW (tv), GTK_WRAP_WORD_CHAR);
      gtk_scrolled_window_set_child (GTK_SCROLLED_WINDOW (scr), tv);

      tfe_text_view_set_file (TFE_TEXT_VIEW (tv),  g_file_dup (files[i]));
      gtk_text_buffer_set_text (tb, contents, length);
      g_free (contents);
      filename = g_file_get_basename (files[i]);
      lab = gtk_label_new (filename);
      gtk_notebook_append_page (GTK_NOTEBOOK (nb), scr, lab);
      nbp = gtk_notebook_get_page (GTK_NOTEBOOK (nb), scr);
      g_object_set (nbp, "tab-expand", TRUE, NULL);
      g_free (filename);
    } else {
      g_printerr ("%s.\n", err->message);
      g_clear_error (&err);
    }
  }
  if (gtk_notebook_get_n_pages (GTK_NOTEBOOK (nb)) > 0) {
    gtk_window_present (GTK_WINDOW (win));
  } else
    gtk_window_destroy (GTK_WINDOW (win));
}
\end{lstlisting}

The function \passthrough{\lstinline!app\_open!} builds the widgets in
the main application window.

\begin{itemize}
\tightlist
\item
  26-28: Creates a GtkApplicationWindow instance and sets the title and
  default size.
\item
  30-31: Creates a GtkBox instance \passthrough{\lstinline!boxv!}. It is
  a vertical box and a child of GtkApplicationWindow. It has two
  children. The first child is a horizontal box. The second child is a
  GtkNotebook.
\item
  33-34: Creates a GtkBox instance \passthrough{\lstinline!boxh!} and
  appends it to \passthrough{\lstinline!boxv!} as the first child.
\item
  36-41: Creates three dummy labels. The labels
  \passthrough{\lstinline!dmy1!} and \passthrough{\lstinline!dmy3!} has
  a character width of ten. The other label
  \passthrough{\lstinline!dmy2!} has hexpand property which is set to be
  TRUE. This makes the label expands horizontally as long as possible.
\item
  42-45: Creates four buttons.
\item
  47-53: Appends these GtkLabel and GtkButton to
  \passthrough{\lstinline!boxh!}.
\item
  55-58: Creates a GtkNotebook instance and sets hexpand and vexpand
  properties to be TRUE. This makes it expand horizontally and
  vertically as big as possible. It is appended to
  \passthrough{\lstinline!boxv!} as the second child.
\end{itemize}

The number of widget-build lines is 33(=58-26+1). We also needed many
variables (\passthrough{\lstinline!boxv!},
\passthrough{\lstinline!boxh!}, \passthrough{\lstinline!dmy1!}, \ldots)
and most of them used only for building the widgets. Are there any good
solution to reduce these works?

Gtk provides GtkBuilder. It reads user interface (UI) data and builds a
window. It reduces this cumbersome work.

\subsection{The UI File}\label{the-ui-file}

Look at the UI file \passthrough{\lstinline!tfe3.ui!} that defines
widget structure.

\begin{lstlisting}[language=XML, numbers=left]
<?xml version="1.0" encoding="UTF-8"?>
<interface>
  <object class="GtkApplicationWindow" id="win">
    <property name="title">file editor</property>
    <property name="default-width">600</property>
    <property name="default-height">400</property>
    <child>
      <object class="GtkBox">
        <property name="orientation">GTK_ORIENTATION_VERTICAL</property>
        <child>
          <object class="GtkBox">
            <property name="orientation">GTK_ORIENTATION_HORIZONTAL</property>
            <child>
              <object class="GtkLabel">
                <property name="width-chars">10</property>
              </object>
            </child>
            <child>
              <object class="GtkButton">
                <property name="label">New</property>
              </object>
            </child>
            <child>
              <object class="GtkButton">
                <property name="label">Open</property>
              </object>
            </child>
            <child>
              <object class="GtkLabel">
                <property name="hexpand">TRUE</property>
              </object>
            </child>
            <child>
              <object class="GtkButton">
                <property name="label">Save</property>
              </object>
            </child>
            <child>
              <object class="GtkButton">
                <property name="label">Close</property>
              </object>
            </child>
            <child>
              <object class="GtkLabel">
                <property name="width-chars">10</property>
              </object>
            </child>
          </object>
        </child>
        <child>
          <object class="GtkNotebook" id="nb">
            <property name="hexpand">TRUE</property>
            <property name="vexpand">TRUE</property>
          </object>
        </child>
      </object>
    </child>
  </object>
</interface>
\end{lstlisting}

The is a XML file. Tags begin with \passthrough{\lstinline!<!} and end
with \passthrough{\lstinline!>!}. There are two types of tags, the start
tag and the end tag. For example, \passthrough{\lstinline!<interface>!}
is a start tag and \passthrough{\lstinline!</interface>!} is an end tag.
The UI file begins and ends with interface tags. Some tags, for example
object tags, can have a class and id attributes in their start tag.

\begin{itemize}
\tightlist
\item
  1: XML declaration. It specifies that the XML version is 1.0 and the
  encoding is UTF-8.
\item
  3-6: An object tag with \passthrough{\lstinline!GtkApplicationWindow!}
  class and \passthrough{\lstinline!win!} id. This is the top level
  window. And the three properties of the window are defined. The
  \passthrough{\lstinline!title!} property is ``file editor'',
  \passthrough{\lstinline!default-width!} property is 600 and
  \passthrough{\lstinline!default-height!} property is 400.
\item
  7: Child tag means a child widget. For example, line 7 tells us that
  GtkBox object is a child widget of \passthrough{\lstinline!win!}.
\end{itemize}

Compare this ui file and the lines 26-58 in the
\passthrough{\lstinline!app\_open!} function of
\passthrough{\lstinline!tfe2.c!}. Both builds the same window with its
descendant widgets.

You can check the ui file with
\passthrough{\lstinline!gtk4-builder-tool!}.

\begin{itemize}
\tightlist
\item
  \passthrough{\lstinline!gtk4-builder-tool validate <ui file name>!}
  validates the ui file. If the ui file includes some syntactical error,
  \passthrough{\lstinline!gtk4-builder-tool!} prints the error.
\item
  \passthrough{\lstinline!gtk4-builder-tool simplify <ui file name>!}
  simplifies the ui file and prints the result. If
  \passthrough{\lstinline!--replace!} option is given, it replaces the
  ui file with the simplified one. If the ui file specifies a value of
  property but it is default, then it will be removed. For example, the
  default orientation is horizontal so the simplification removes line
  12. And some values are simplified. For example, ``TRUE''and ``FALSE''
  becomes ``1'' and ``0'' respectively. However, ``TRUE'' or ``FALSE''
  is better for maintenance.
\end{itemize}

It is a good idea to check your ui file before compiling.

\subsection{GtkBuilder}\label{gtkbuilder}

GtkBuilder builds widgets based on a ui file.

\begin{lstlisting}[language=C]
GtkBuilder *build;

build = gtk_builder_new_from_file ("tfe3.ui");
win = GTK_WIDGET (gtk_builder_get_object (build, "win"));
gtk_window_set_application (GTK_WINDOW (win), GTK_APPLICATION (app));
nb = GTK_WIDGET (gtk_builder_get_object (build, "nb"));
g_object_unref(build);
\end{lstlisting}

The function \passthrough{\lstinline!gtk\_builder\_new\_from\_file!}
reads the file \passthrough{\lstinline!tfe3.ui!}. Then, it builds the
widgets and creates GtkBuilder object. All the widgets are connected
based on the parent-children relationship described in the ui file. We
can retrieve objects from the builder object with
\passthrough{\lstinline!gtk\_builder\_get\_object!} function. The top
level window, its id is ``win'' in the ui file, is taken and assigned to
the variable \passthrough{\lstinline!win!}, the application property of
which is set to \passthrough{\lstinline!app!} with the
\passthrough{\lstinline!gtk\_window\_set\_application!} function.
GtkNotebook with the id ``nb'' in the ui file is also taken and assigned
to the variable \passthrough{\lstinline!nb!}. After the window and
application are connected, GtkBuilder instance is useless. It is
released with \passthrough{\lstinline!g\_object\_unref!} function.

The ui file reduces lines in the C source file.

\begin{lstlisting}
$ cd tfe; diff tfe2.c tfe3.c
59a60
>   GtkBuilder *build;
61,104c62,66
<   GtkWidget *boxv;
<   GtkWidget *boxh;
<   GtkWidget *dmy1;
<   GtkWidget *dmy2;
<   GtkWidget *dmy3;
<   GtkWidget *btnn; /* button for new */
<   GtkWidget *btno; /* button for open */
<   GtkWidget *btns; /* button for save */
<   GtkWidget *btnc; /* button for close */
< 
<   win = gtk_application_window_new (GTK_APPLICATION (app));
<   gtk_window_set_title (GTK_WINDOW (win), "file editor");
<   gtk_window_set_default_size (GTK_WINDOW (win), 600, 400);
< 
<   boxv = gtk_box_new (GTK_ORIENTATION_VERTICAL, 0);
<   gtk_window_set_child (GTK_WINDOW (win), boxv);
< 
<   boxh = gtk_box_new (GTK_ORIENTATION_HORIZONTAL, 0);
<   gtk_box_append (GTK_BOX (boxv), boxh);
< 
<   dmy1 = gtk_label_new(NULL); /* dummy label for left space */
<   gtk_label_set_width_chars (GTK_LABEL (dmy1), 10);
<   dmy2 = gtk_label_new(NULL); /* dummy label for center space */
<   gtk_widget_set_hexpand (dmy2, TRUE);
<   dmy3 = gtk_label_new(NULL); /* dummy label for right space */
<   gtk_label_set_width_chars (GTK_LABEL (dmy3), 10);
<   btnn = gtk_button_new_with_label ("New");
<   btno = gtk_button_new_with_label ("Open");
<   btns = gtk_button_new_with_label ("Save");
<   btnc = gtk_button_new_with_label ("Close");
< 
<   gtk_box_append (GTK_BOX (boxh), dmy1);
<   gtk_box_append (GTK_BOX (boxh), btnn);
<   gtk_box_append (GTK_BOX (boxh), btno);
<   gtk_box_append (GTK_BOX (boxh), dmy2);
<   gtk_box_append (GTK_BOX (boxh), btns);
<   gtk_box_append (GTK_BOX (boxh), btnc);
<   gtk_box_append (GTK_BOX (boxh), dmy3);
< 
<   nb = gtk_notebook_new ();
<   gtk_widget_set_hexpand (nb, TRUE);
<   gtk_widget_set_vexpand (nb, TRUE);
<   gtk_box_append (GTK_BOX (boxv), nb);
< 
---
>   build = gtk_builder_new_from_file ("tfe3.ui");
>   win = GTK_WIDGET (gtk_builder_get_object (build, "win"));
>   gtk_window_set_application (GTK_WINDOW (win), GTK_APPLICATION (app));
>   nb = GTK_WIDGET (gtk_builder_get_object (build, "nb"));
>   g_object_unref(build);
138c100
<   app = gtk_application_new ("com.github.ToshioCP.tfe2", G_APPLICATION_HANDLES_OPEN);
---
>   app = gtk_application_new ("com.github.ToshioCP.tfe3", G_APPLICATION_HANDLES_OPEN);
144a107
> 
\end{lstlisting}

\passthrough{\lstinline!61,104c62,66!} means that 44 (=104-61+1) lines
are changed to 5 (=66-62+1) lines. Therefore, 39 lines are reduced.
Using ui file not only shortens C source files, but also makes the
widgets structure clear.

Now I'll show you \passthrough{\lstinline!app\_open!} function in the C
file \passthrough{\lstinline!tfe3.c!}.

\begin{lstlisting}[language=C, numbers=left]
static void
app_open (GApplication *app, GFile ** files, gint n_files, gchar *hint) {
  GtkWidget *win;
  GtkWidget *nb;
  GtkWidget *lab;
  GtkNotebookPage *nbp;
  GtkWidget *scr;
  GtkWidget *tv;
  GtkTextBuffer *tb;
  char *contents;
  gsize length;
  char *filename;
  int i;
  GError *err = NULL;
  GtkBuilder *build;

  build = gtk_builder_new_from_file ("tfe3.ui");
  win = GTK_WIDGET (gtk_builder_get_object (build, "win"));
  gtk_window_set_application (GTK_WINDOW (win), GTK_APPLICATION (app));
  nb = GTK_WIDGET (gtk_builder_get_object (build, "nb"));
  g_object_unref(build);
  for (i = 0; i < n_files; i++) {
    if (g_file_load_contents (files[i], NULL, &contents, &length, NULL, &err)) {
      scr = gtk_scrolled_window_new ();
      tv = tfe_text_view_new ();
      tb = gtk_text_view_get_buffer (GTK_TEXT_VIEW (tv));
      gtk_text_view_set_wrap_mode (GTK_TEXT_VIEW (tv), GTK_WRAP_WORD_CHAR);
      gtk_scrolled_window_set_child (GTK_SCROLLED_WINDOW (scr), tv);

      tfe_text_view_set_file (TFE_TEXT_VIEW (tv),  g_file_dup (files[i]));
      gtk_text_buffer_set_text (tb, contents, length);
      g_free (contents);
      filename = g_file_get_basename (files[i]);
      lab = gtk_label_new (filename);
      gtk_notebook_append_page (GTK_NOTEBOOK (nb), scr, lab);
      nbp = gtk_notebook_get_page (GTK_NOTEBOOK (nb), scr);
      g_object_set (nbp, "tab-expand", TRUE, NULL);
      g_free (filename);
    } else {
      g_printerr ("%s.\n", err->message);
      g_clear_error (&err);
    }
  }
  if (gtk_notebook_get_n_pages (GTK_NOTEBOOK (nb)) > 0) {
    gtk_window_present (GTK_WINDOW (win));
  } else
    gtk_window_destroy (GTK_WINDOW (win));
}
\end{lstlisting}

The whole source code of \passthrough{\lstinline!tfe3.c!} is stored in
the src/tfe directory.

\subsubsection{Using ui string}\label{using-ui-string}

GtkBuilder can build widgets with string. Use
\passthrough{\lstinline!gtk\_builder\_new\_from\_string!} instead of
\passthrough{\lstinline!gtk\_builder\_new\_from\_file!}.

\begin{lstlisting}[language=C]
char *uistring;

uistring =
"<interface>"
  "<object class="GtkApplicationWindow" id="win">"
    "<property name=\"title\">file editor</property>"
    "<property name=\"default-width\">600</property>"
    "<property name=\"default-height\">400</property>"
    "<child>"
      "<object class=\"GtkBox\">"
        "<property name="orientation">GTK_ORIENTATION_VERTICAL</property>"
... ... ...
... ... ...
"</interface>";

build = gtk_builder_new_from_string (uistring, -1);
\end{lstlisting}

This method has an advantage and disadvantage. The advantage is that the
ui string is written in the source code. So, no ui file is needed on
runtime. The disadvantage is that writing C string is a bit bothersome
because of the double quotes. If you want to use this method, you should
write a script that transforms ui file into C-string.

\begin{itemize}
\tightlist
\item
  Add backslash before each double quote.
\item
  Add double quotes at the left and right of the string in each line.
\end{itemize}

\subsubsection{Gresource}\label{gresource}

Gresource is similar to string. But Gresource is compressed binary data,
not text data. And there's a compiler that compiles ui file into
Gresource. It can compile not only text files but also binary files such
as images, sounds and so on. And after compilation, it bundles them up
into one Gresource object.

An xml file is necessary for the resource compiler
\passthrough{\lstinline!glib-compile-resources!}. It describes resource
files.

\begin{lstlisting}[language=XML, numbers=left]
<?xml version="1.0" encoding="UTF-8"?>
<gresources>
  <gresource prefix="/com/github/ToshioCP/tfe3">
    <file>tfe3.ui</file>
  </gresource>
</gresources>
\end{lstlisting}

\begin{itemize}
\tightlist
\item
  2: \passthrough{\lstinline!gresources!} tag can include multiple
  gresources (gresource tags). However, this xml has only one gresource.
\item
  3: The gresource has a prefix
  \passthrough{\lstinline!/com/github/ToshioCP/tfe3!}.
\item
  4: The name of the gresource is \passthrough{\lstinline!tfe3.ui!}. The
  resource will be pointed with
  \passthrough{\lstinline!/com/github/ToshioCP/tfe3/tfe3.ui!} by
  GtkBuilder. The pattern is ``prefix'' + ``name''. If you want to add
  more files, insert them between line 4 and 5.
\end{itemize}

Save this xml text to \passthrough{\lstinline!tfe3.gresource.xml!}. The
gresource compiler \passthrough{\lstinline!glib-compile-resources!}
shows its usage with the argument \passthrough{\lstinline!--help!}.

\begin{lstlisting}
$ glib-compile-resources --help
Usage:
  glib-compile-resources [OPTION..] FILE

Compile a resource specification into a resource file.
Resource specification files have the extension .gresource.xml,
and the resource file have the extension called .gresource.

Help Options:
  -h, --help                   Show help options

Application Options:
  --version                    Show program version and exit
  --target=FILE                Name of the output file
  --sourcedir=DIRECTORY        The directories to load files referenced in FILE from (default: current directory)
  --generate                   Generate output in the format selected for by the target filename extension
  --generate-header            Generate source header
  --generate-source            Generate source code used to link in the resource file into your code
  --generate-dependencies      Generate dependency list
  --dependency-file=FILE       Name of the dependency file to generate
  --generate-phony-targets     Include phony targets in the generated dependency file
  --manual-register            Don't automatically create and register resource
  --internal                   Don't export functions; declare them G_GNUC_INTERNAL
  --external-data              Don't embed resource data in the C file; assume it's linked externally instead
  --c-name                     C identifier name used for the generated source code
  -C, --compiler               The target C compiler (default: the CC environment variable)
\end{lstlisting}

Now run the compiler.

\begin{lstlisting}
$ glib-compile-resources tfe3.gresource.xml --target=resources.c --generate-source
\end{lstlisting}

Then a C source file \passthrough{\lstinline!resources.c!} is generated.
Modify \passthrough{\lstinline!tfe3.c!} and save it as
\passthrough{\lstinline!tfe3\_r.c!}.

\begin{lstlisting}[language=C]
#include "resources.c"
... ... ...
... ... ...
build = gtk_builder_new_from_resource ("/com/github/ToshioCP/tfe3/tfe3.ui");
... ... ...
... ... ...
\end{lstlisting}

The function \passthrough{\lstinline!gtk\_builder\_new\_from\_resource!}
builds widgets from a resource.

Then, compile and run it.

\begin{lstlisting}
$ comp tfe3_r
$ ./a.out tfe2.c
\end{lstlisting}

A window appears and it is the same as the screenshot at the beginning
of this page.

Generally, resource is the best way for C language. If you use other
languages like Ruby, string is better than resource.
