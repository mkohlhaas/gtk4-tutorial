\section{GtkGridView and activate
signal}\label{gtkgridview-and-activate-signal}

GtkGridView is similar to GtkListView. It displays a GListModel as a
grid, which is like a square tessellation.

\begin{figure}
\centering
\includegraphics[width=10cm,height=6.6cm]{../image/list4.png}
\caption{Grid}
\end{figure}

This is often seen when you use a file browser like GNOME Files
(Nautilus).

In this section, let's make a very simple file browser
\passthrough{\lstinline!list4!}. It just shows the files in the current
directory. And a user can choose list or grid by clicking on buttons in
the tool bar. Each item in the list or grid has an icon and a filename.
In addition, \passthrough{\lstinline!list4!} provides the way to open
the \passthrough{\lstinline!tfe!} text editor to show a text file. A
user can do that by double clicking on an item or pressing enter key
when an item is selected.

\subsection{GtkDirectoryList}\label{gtkdirectorylist}

GtkDirectoryList implements GListModel and it contains information of
files in a certain directory. The items of the list are GFileInfo
objects.

In the \passthrough{\lstinline!list4!} source files, GtkDirectoryList is
described in a ui file and built by GtkBuilder. The GtkDirectoryList
instance is assigned to the ``model'' property of a GtkSingleSelection
instance. And the GtkSingleSelection instance is assigned to the
``model'' property of a GListView or GGridView instance.

\begin{lstlisting}
GtkListView (model property) => GtkSingleSelection (model property) => GtkDirectoryList
GtkGridView (model property) => GtkSingleSelection (model property) => GtkDirectoryList
\end{lstlisting}

\begin{figure}
\centering
\includegraphics[width=10cm,height=7.5cm]{../image/directorylist.png}
\caption{DirectoryList}
\end{figure}

The following is a part of the ui file
\passthrough{\lstinline!list4.ui!}.

\begin{lstlisting}[language=XML]
<object class="GtkListView" id="list">
  <property name="model">
    <object class="GtkSingleSelection" id="singleselection">
      <property name="model">
        <object class="GtkDirectoryList" id="directorylist">
          <property name="attributes">standard::name,standard::icon,standard::content-type</property>
        </object>
      </property>
    </object>
  </property>
</object>
<object class="GtkGridView" id="grid">
  <property name="model">singleselection</property>
</object>
\end{lstlisting}

GtkDirectoryList has an ``attributes'' property. It is attributes of
GFileInfo such as ``standard::name'', ``standard::icon'' and
``standard::content-type''.

\begin{itemize}
\tightlist
\item
  standard::name is a filename.
\item
  standard::icon is an icon of the file. It is a GIcon object.
\item
  standard::content-type is a content-type. Content-type is the same as
  mime type for the internet. For example, ``text/plain'' is a text
  file, ``text/x-csrc'' is a C source code and so on. (``text/x-csrc''is
  not registered to IANA media types. Such ``x-'' subtype is not a
  standard mime type.) Content type is also used by desktop systems.
\end{itemize}

GtkGridView uses the same GtkSingleSelection instance
(\passthrough{\lstinline!singleselection!}). So, its model property is
set to it.

\subsection{Ui file of the window}\label{ui-file-of-the-window}

The window is built with the following ui file. (See the screenshot at
the beginning of this section).

\begin{lstlisting}[language=XML, numbers=left]
<?xml version="1.0" encoding="UTF-8"?>
<interface>
  <object class="GtkApplicationWindow" id="win">
    <property name="title">file list</property>
    <property name="default-width">600</property>
    <property name="default-height">400</property>
    <child>
      <object class="GtkBox" id="boxv">
        <property name="orientation">GTK_ORIENTATION_VERTICAL</property>
        <child>
          <object class="GtkBox" id="boxh">
            <property name="orientation">GTK_ORIENTATION_HORIZONTAL</property>
            <child>
              <object class="GtkLabel" id="dmy1">
                <property name="hexpand">TRUE</property>
              </object>
            </child>
            <child>
              <object class="GtkButton" id="btnlist">
                <property name="name">btnlist</property>
                <property name="action-name">win.view</property>
                <property name="action-target">&apos;list&apos;</property>
                <child>
                  <object class="GtkImage">
                    <property name="resource">/com/github/ToshioCP/list4/list.png</property>
                  </object>
                </child>
              </object>
            </child>
            <child>
              <object class="GtkButton" id="btngrid">
                <property name="name">btngrid</property>
                <property name="action-name">win.view</property>
                <property name="action-target">&apos;grid&apos;</property>
                <child>
                  <object class="GtkImage">
                    <property name="resource">/com/github/ToshioCP/list4/grid.png</property>
                  </object>
                </child>
              </object>
            </child>
            <child>
              <object class="GtkLabel" id="dmy2">
                <property name="width-chars">10</property>
              </object>
            </child>
          </object>
        </child>
        <child>
          <object class="GtkScrolledWindow" id="scr">
            <property name="hexpand">TRUE</property>
            <property name="vexpand">TRUE</property>
          </object>
        </child>
      </object>
    </child>
  </object>
  <object class="GtkListView" id="list">
    <property name="model">
      <object class="GtkSingleSelection" id="singleselection">
        <property name="model">
          <object class="GtkDirectoryList" id="directory_list">
            <property name="attributes">standard::name,standard::icon,standard::content-type</property>
          </object>
        </property>
      </object>
    </property>
  </object>
  <object class="GtkGridView" id="grid">
    <property name="model">singleselection</property>
  </object>
</interface>
\end{lstlisting}

The file consists of two parts. The first part begins at the line 3 and
ends at line 57. This part is the widgets from the top level window to
the scrolled window. It also includes two buttons. The second part
begins at line 58 and ends at line 71. This is the part of GtkListView
and GtkGridView.

\begin{itemize}
\tightlist
\item
  13-17, 42-46: Two labels are dummy labels. They just work as a space
  to put the two buttons at the appropriate position.
\item
  18-41: GtkButton \passthrough{\lstinline!btnlist!} and
  \passthrough{\lstinline!btngrid!}. These two buttons work as selection
  buttons to switch from list to grid and vice versa. These two buttons
  are connected to a stateful action \passthrough{\lstinline!win.view!}.
  This action has a parameter. Such action consists of prefix, action
  name and parameter. The prefix of the action is
  \passthrough{\lstinline!win!}, which means the action belongs to the
  top level window. The prefix gives the scope of the action. The action
  name is \passthrough{\lstinline!view!}. The parameters are
  \passthrough{\lstinline!list!} or \passthrough{\lstinline!grid!},
  which show the state of the action. A parameter is also called a
  target, because it is a target to which the action changes its state.
  We often write the detailed action like ``win.view::list'' or
  ``win.view::grid''.
\item
  21-22: The properties ``action-name'' and ``action-target'' belong to
  GtkActionable interface. GtkButton implements GtkActionable. The
  action name is ``win.view'' and the target is ``list''. Generally, a
  target is GVariant, which can be string, integer, float and so on. You
  need to use GVariant text format to write GVariant value in ui files.
  If the type of the GVariant value is string, then the value with
  GVariant text format is bounded by single quotes or double quotes.
  Because ui file is xml format text, single quote cannot be written
  without escape. Its escape sequence is \&apos;. Therefore, the target
  `list' is written as \&apos;list\&apos;. Because the button is
  connected to the action, ``clicked'' signal handler isn't needed.
\item
  23-27: The child widget of the button is GtkImage. GtkImage has a
  ``resource'' property. It is a GResource and GtkImage reads an image
  data from the resource and sets the image. This resource is built from
  24x24-sized png image data, which is an original icon.
\item
  50-53: GtkScrolledWindow. Its child widget will be set with
  GtkListView or GtkGridView.
\end{itemize}

The action \passthrough{\lstinline!view!} is created, connected to the
``activate'' signal handler and inserted to the window (action map) as
follows.

\begin{lstlisting}[language=C]
act_view = g_simple_action_new_stateful ("view", g_variant_type_new("s"), g_variant_new_string ("list"));
g_signal_connect (act_view, "activate", G_CALLBACK (view_activated), NULL);
g_action_map_add_action (G_ACTION_MAP (win), G_ACTION (act_view));
\end{lstlisting}

The signal handler \passthrough{\lstinline!view\_activated!} will be
explained later.

\subsection{Factories}\label{factories}

Each view (GtkListView and GtkGridView) has its own factory because its
items have different structure of widgets. The factories are
GtkBuilderListItemFactory objects. Their ui files are as follows.

factory\_list.ui

\begin{lstlisting}[language=XML, numbers=left]
<?xml version="1.0" encoding="UTF-8"?>
<interface>
  <template class="GtkListItem">
    <property name="child">
      <object class="GtkBox">
        <property name="orientation">GTK_ORIENTATION_HORIZONTAL</property>
        <property name="spacing">20</property>
        <child>
          <object class="GtkImage">
            <binding name="gicon">
              <closure type="GIcon" function="get_icon">
                <lookup name="item">GtkListItem</lookup>
              </closure>
            </binding>
          </object>
        </child>
        <child>
          <object class="GtkLabel">
            <property name="hexpand">TRUE</property>
            <property name="xalign">0</property>
            <binding name="label">
              <closure type="gchararray" function="get_file_name">
                <lookup name="item">GtkListItem</lookup>
              </closure>
            </binding>
          </object>
        </child>
      </object>
    </property>
  </template>
</interface>
\end{lstlisting}

factory\_grid.ui

\begin{lstlisting}[language=XML, numbers=left]
<?xml version="1.0" encoding="UTF-8"?>
<interface>
  <template class="GtkListItem">
    <property name="child">
      <object class="GtkBox">
        <property name="orientation">GTK_ORIENTATION_VERTICAL</property>
        <property name="spacing">20</property>
        <child>
          <object class="GtkImage">
            <property name="icon-size">GTK_ICON_SIZE_LARGE</property>
            <binding name="gicon">
              <closure type="GIcon" function="get_icon">
                <lookup name="item">GtkListItem</lookup>
              </closure>
            </binding>
          </object>
        </child>
        <child>
          <object class="GtkLabel">
            <property name="hexpand">TRUE</property>
            <property name="xalign">0.5</property>
            <binding name="label">
              <closure type="gchararray" function="get_file_name">
                <lookup name="item">GtkListItem</lookup>
              </closure>
            </binding>
          </object>
        </child>
      </object>
    </property>
  </template>
</interface>
\end{lstlisting}

The two files above are almost same. The difference is:

\begin{itemize}
\tightlist
\item
  The orientation of the box
\item
  The icon size
\item
  The position of the text of the label
\end{itemize}

\begin{lstlisting}
$ cd list4; diff factory_list.ui factory_grid.ui
6c6
<         <property name="orientation">GTK_ORIENTATION_HORIZONTAL</property>
---
>         <property name="orientation">GTK_ORIENTATION_VERTICAL</property>
9a10
>             <property name="icon-size">GTK_ICON_SIZE_LARGE</property>
20c21
<             <property name="xalign">0</property>
---
>             <property name="xalign">0.5</property>
\end{lstlisting}

Two properties ``gicon'' (property of GtkImage) and ``label'' (property
of GtkLabel) are in the ui files above. Because GFileInfo doesn't have
properties correspond to icon or filename, the factory uses closure tag
to bind ``gicon'' and ``label'' properties to GFileInfo information. A
function \passthrough{\lstinline!get\_icon!} gets GIcon from the
GFileInfo object. And a function
\passthrough{\lstinline!get\_file\_name!} gets a filename from the
GFileInfo object.

\begin{lstlisting}[language=C, numbers=left]
GIcon *
get_icon (GtkListItem *item, GFileInfo *info) {
  GIcon *icon;

   /* g_file_info_get_icon can return NULL */
  icon = G_IS_FILE_INFO (info) ? g_file_info_get_icon (info) : NULL;
  return icon ? g_object_ref (icon) : NULL;
}

char *
get_file_name (GtkListItem *item, GFileInfo *info) {
  return G_IS_FILE_INFO (info) ? g_strdup (g_file_info_get_name (info)) : NULL;
}
\end{lstlisting}

One important thing is the ownership of the return values. The return
value is owned by the caller. So,
\passthrough{\lstinline!g\_obect\_ref!} or
\passthrough{\lstinline!g\_strdup!} is necessary.

\subsection{An activate signal handler of the button
action}\label{an-activate-signal-handler-of-the-button-action}

An activate signal handler \passthrough{\lstinline!view\_activate!}
switches the view. It does two things.

\begin{itemize}
\tightlist
\item
  Changes the child widget of GtkScrolledWindow.
\item
  Changes the CSS of buttons to show the current state.
\end{itemize}

\begin{lstlisting}[language=C, numbers=left]
static void
view_activated(GSimpleAction *action, GVariant *parameter) {
  const char *view = g_variant_get_string (parameter, NULL);
  const char *other;
  char *css;

  if (strcmp (view, "list") == 0) {
    other = "grid";
    gtk_scrolled_window_set_child (scr, GTK_WIDGET (list));
  }else {
    other = "list";
    gtk_scrolled_window_set_child (scr, GTK_WIDGET (grid));
  }
  css = g_strdup_printf ("button#btn%s {background: silver;} button#btn%s {background: white;}", view, other);
  gtk_css_provider_load_from_data (provider, css, -1);
  g_free (css);
  g_action_change_state (G_ACTION (action), parameter);
}
\end{lstlisting}

The second parameter of this handler is the target of the clicked
button. Its type is GVariant.

\begin{itemize}
\tightlist
\item
  If \passthrough{\lstinline!btnlist!} has been clicked, then
  \passthrough{\lstinline!parameter!} is a GVariant of the string
  ``list''.
\item
  If \passthrough{\lstinline!btngrid!} has been clicked, then
  \passthrough{\lstinline!parameter!} is a GVariant of the string
  ``grid''.
\end{itemize}

The third parameter \passthrough{\lstinline!user\_data!} points NULL and
it is ignored here.

\begin{itemize}
\tightlist
\item
  3: \passthrough{\lstinline!g\_variant\_get\_string!} gets the string
  from the GVariant variable.
\item
  7-13: Sets the child of \passthrough{\lstinline!scr!}. The function
  \passthrough{\lstinline!gtk\_scrolled\_window\_set\_child!} decreases
  the reference count of the old child by one. And it increases the
  reference count of the new child by one.
\item
  14-16: Sets the CSS for the buttons. The background of the clicked
  button will be silver color and the other button will be white.
\item
  17: Changes the state of the action.
\end{itemize}

\subsection{Activate signal on GtkListView and
GtkGridView}\label{activate-signal-on-gtklistview-and-gtkgridview}

Views (GtkListView and GtkGridView) have an ``activate'' signal. It is
emitted when an item in the view is double clicked or the enter key is
pressed. You can do anything you like by connecting the ``activate''
signal to the handler.

The example \passthrough{\lstinline!list4!} launches
\passthrough{\lstinline!tfe!} text file editor if the item of the list
is a text file.

\begin{lstlisting}[language=C]
static void
list_activate (GtkListView *list, int position, gpointer user_data) {
  GFileInfo *info = G_FILE_INFO (g_list_model_get_item (G_LIST_MODEL (gtk_list_view_get_model (list)), position));
  launch_tfe_with_file (info);
}

static void
grid_activate (GtkGridView *grid, int position, gpointer user_data) {
  GFileInfo *info = G_FILE_INFO (g_list_model_get_item (G_LIST_MODEL (gtk_grid_view_get_model (grid)), position));
  launch_tfe_with_file (info);
}

... ...
... ...

  g_signal_connect (GTK_LIST_VIEW (list), "activate", G_CALLBACK (list_activate), NULL);
  g_signal_connect (GTK_GRID_VIEW (grid), "activate", G_CALLBACK (grid_activate), NULL);
\end{lstlisting}

The second parameter of each handler is the position of the item
(GFileInfo) of the GListModel. So you can get the item with
\passthrough{\lstinline!g\_list\_model\_get\_item!} function.

\subsubsection{Content type and application
launch}\label{content-type-and-application-launch}

The function \passthrough{\lstinline!launch\_tfe\_with\_file!} gets a
file from the GFileInfo instance. If the file is a text file, it
launches \passthrough{\lstinline!tfe!} with the file.

GFileInfo has information about file type. The file type is like
``text/plain'', ``text/x-csrc'' and so on. It is called content type.
Content type can be got with
\passthrough{\lstinline!g\_file\_info\_get\_content\_type!} function.

\begin{lstlisting}[language=C, numbers=left]
static void
launch_tfe_with_file (GFileInfo *info) {
  GError *err = NULL;
  GFile *file;
  GList *files = NULL;
  const char *content_type;
  const char *text_type = "text/";
  GAppInfo *appinfo;
  int i;

  if (! info)
    return;
  content_type = g_file_info_get_content_type (info);
#ifdef debug
  g_print ("%s\n", content_type);
#endif
  if (! content_type)
    return;
  for (i=0;i<5;++i) { /* compare the first 5 characters */
    if (content_type[i] != text_type[i])
      return;
  }
  appinfo = g_app_info_create_from_commandline ("tfe", "tfe", G_APP_INFO_CREATE_NONE, &err);
  if (err) {
    g_printerr ("%s\n", err->message);
    g_error_free (err);
    return;
  }
  err = NULL;
  file = g_file_new_for_path (g_file_info_get_name (info));
  files = g_list_append (files, file);
  if (! (g_app_info_launch (appinfo, files, NULL, &err))) {
    g_printerr ("%s\n", err->message);
    g_error_free (err);
  }
  g_list_free_full (files, g_object_unref);
  g_object_unref (appinfo);
}
\end{lstlisting}

\begin{itemize}
\tightlist
\item
  13: Gets the content type of the file from GFileInfo.
\item
  14-16: Prints the content type if ``debug'' is defined. This is only
  useful to know a content type of a file. If you don't want this,
  delete or uncomment the definition
  \passthrough{\lstinline!\#define debug 1!} iat line 6 in the source
  file.
\item
  17-22: If no content type or the content type doesn't begin with
  ``text/'',the function returns.
\item
  23: Creates GAppInfo object of \passthrough{\lstinline!tfe!}
  application. GAppInfo is an interface and the variable
  \passthrough{\lstinline!appinfo!} points a GDesktopAppInfo instance.
  GAppInfo is a collection of information of applications.
\item
  32: Launches the application (\passthrough{\lstinline!tfe!}) with an
  argument \passthrough{\lstinline!file!}.
  \passthrough{\lstinline!g\_app\_info\_launch!} has four parameters.
  The first parameter is GAppInfo object. The second parameter is a list
  of GFile objects. In this function, only one GFile instance is given
  to \passthrough{\lstinline!tfe!}, but you can give more arguments. The
  third parameter is GAppLaunchContext, but this program gives NULL
  instead. The last parameter is the pointer to the pointer to a GError.
\item
  36: \passthrough{\lstinline!g\_list\_free\_full!} frees the memories
  used by the list and items.
\end{itemize}

If your distribution supports GTK 4, using
\passthrough{\lstinline!g\_app\_info\_launch\_default\_for\_uri!} is
convenient. The function automatically determines the default
application from the file and launches it. For example, if the file is
text, then it launches GNOME text editor with the file. Such feature
comes from desktop.

\subsection{Compilation and execution}\label{compilation-and-execution}

The source files are located in src/list4 directory. To compile and
execute list4, type as follows.

\begin{lstlisting}
$ cd list4 # or cd src/list4. It depends your current directory.
$ meson setup _build
$ ninja -C _build
$ _build/list4
\end{lstlisting}

Then a file list appears as a list style. Click on a button on the tool
bar so that you can change the style to grid or back to list. Double
click ``list4.c'' item, then \passthrough{\lstinline!tfe!} text editor
runs with the argument ``list4.c''. The following is the screenshot.

\begin{figure}
\centering
\includegraphics[width=8cm,height=5.62cm]{../image/screenshot_list4.png}
\caption{Screenshot}
\end{figure}

\subsection{``gbytes'' property of
GtkBuilderListItemFactory}\label{gbytes-property-of-gtkbuilderlistitemfactory}

GtkBuilderListItemFactory has ``gbytes'' property. The property contains
a byte sequence of ui data. If you use this property, you can put the
contents of \passthrough{\lstinline!factory\_list.ui!} and
\passthrough{\lstinline!factory\_grid.ui!}into
\passthrough{\lstinline!list4.ui!}. The following shows a part of the
new ui file (\passthrough{\lstinline!list5.ui!}).

\begin{lstlisting}[language=XML]
  <object class="GtkListView" id="list">
    <property name="model">
      <object class="GtkSingleSelection" id="singleselection">
        <property name="model">
          <object class="GtkDirectoryList" id="directory_list">
            <property name="attributes">standard::name,standard::icon,standard::content-type</property>
          </object>
        </property>
      </object>
    </property>
    <property name="factory">
      <object class="GtkBuilderListItemFactory">
        <property name="bytes"><![CDATA[
<?xml version="1.0" encoding="UTF-8"?>
<interface>
  <template class="GtkListItem">
    <property name="child">
      <object class="GtkBox">
        <property name="orientation">GTK_ORIENTATION_HORIZONTAL</property>
        <property name="spacing">20</property>
        <child>
          <object class="GtkImage">
            <binding name="gicon">
              <closure type="GIcon" function="get_icon">
                <lookup name="item">GtkListItem</lookup>
              </closure>
            </binding>
          </object>
        </child>
        <child>
          <object class="GtkLabel">
            <property name="hexpand">TRUE</property>
            <property name="xalign">0</property>
            <binding name="label">
              <closure type="gchararray" function="get_file_name">
                <lookup name="item">GtkListItem</lookup>
              </closure>
            </binding>
          </object>
        </child>
      </object>
    </property>
  </template>
</interface>
        ]]></property>
      </object>
    </property>
  </object>
\end{lstlisting}

CDATA section begins with ``\textless!{[}CDATA{[}'' and ends with
''{]}{]}\textgreater{}''. The contents of CDATA section is recognized as
a string. Any character, even if it is a key syntax marker such as
`\textless{}' or `\textgreater{}', is recognized literally. Therefore,
the text between ``\textless!{[}CDATA{[}'' and ''{]}{]}\textgreater{}''
is inserted to ``bytes'' property as it is.

This method decreases the number of ui files. But, the new ui file is a
bit complicated especially for the beginners. If you feel some
difficulty, it is better for you to separate the ui file.

A directory src/list5 includes the ui file above.
